\documentclass{article}
\usepackage[utf8]{inputenc}
\usepackage{geometry}
 \geometry{
 a4paper,
 total={170mm,257mm},
 left=20mm,
 top=20mm,
 }


\title{Operator Product Expansion}
\author{Luigi Del Debbio, Joseph Lee, Andrew Yong}
\date{January 2019}

\usepackage{natbib}
\usepackage{graphicx}
\usepackage{amsmath}
\usepackage{textcomp}
\usepackage{empheq}
\usepackage{physics}

\begin{document}

\maketitle
\renewcommand{\abstractname}{\vspace{-\baselineskip}}
\begin{abstract}
These notes aim to supplement Collin's discussion on Operator Product Expansion (Ch.10, Renormalisation\cite{collins_1984}). Useful reference \cite{schwartz}
\end{abstract}

\section{Divergent Example - \S 10.1.2}
In \S10.1, we have been exploring the expansion of the two point function $T\phi(x)\phi(0)$. As explained at the end of \S10.1.1, we will be restricting our attention to the leading-power behaviour, i.e. studying
\begin{align*}
    T\phi(x)\phi(0) \sim C_{\phi^2}(x)[\phi^2]
\end{align*}
where $[\phi^2]$ is the renormalized Green's function. \\

In this `divergent example', we evaluate the correction to $T\phi(x)\phi(0)$. At order $\mathcal{O}(g)$, the correction is given by
\begin{equation}
    \frac{i^2}{(p_1^2-m^2)(p_2^2-m^2)}ig\int \frac{d^4q}{(2\pi)^4} \frac{e^{-iq\cdot x}}{(q^2-m^2)((q-p_1-p_2)^2 - m^2)}.
    \label{loopOg}
\end{equation}
which corresponding feynman diagram is 10.1.3(a).\\

Two regions of momentum $q$ contribute differently to this term:
\begin{enumerate}
    \item $q$ finite as $x \rightarrow 0$, this  provides an $x-$independent correction,
    \item $q$ large (up to $O(1/x)$) as $x \rightarrow 0$, this  provides an $x-$dependent correction.
\end{enumerate}
As we will see, this allows us to express 
\begin{align}
    C_{\phi^2} = 1 + (g/16\pi^2)c_1(x^2)
\end{align}
where the two regions contribute to the two terms respectively. The two terms correspond to the decomposition of fig  10.1.13(a) into (b) and (c), where (b) is simply performing renormalization to $\phi(0)^2$ by cutting off the UV divergence, and (c) takes care of the remaining contribution. 
\subsection{Momentum region 1: $q$ finite, $x\rightarrow 0$}

In the region where $q$ is finite, sending $x\rightarrow 0$ (so $e^{-iq\cdot x} \rightarrow 1$) will get  us the standard logarithmic divergent term of the form 

\begin{equation}
    \frac{i^2}{(p_1^2-m^2)(p_2^2-m^2)}ig\int \frac{d^4q}{(2\pi)^4} \frac{1}{(q^2-m^2)((q-p_1-p_2)^2 - m^2)}.
    \label{logdiv}
\end{equation}

To renormalise this, we can use dimensional regularisation to expose the divergent terms. First, we apply the Feynman parameter trick to Equation \ref{logdiv},

\begin{equation}
    \frac{1}{AB} = \int_0^1 dx \, \frac{1}{(A(1-x) + Bx)^2}.
\label{Feynparam trick}
\end{equation}
Identifying $A=(q^2-m^2)$ and $B=((q-p)^2-m^2)$, with $p=p_1+p_2$, the integral in Equation \ref{logdiv} becomes

\begin{equation}
\begin{split}
    I(p) &= ig\int \frac{d^4q}{(2\pi)^4}\int^1_0 dx \frac{1}{((q^2-m^2)(1-x) +(q-p_1-p_2)^2 - m^2)x)^2},\\
    &= ig\int \frac{d^4q}{(2\pi)^4}\int^1_0 dx \frac{1}{((q-px)^2 + p^2x(1-x) - m^2 )^2},\\
    \mathrm{let}\, q'&= q-px,\\
    &= ig\int \frac{d^4q}{(2\pi)^4}\int^1_0 dx \frac{1}{(q^2 -M^2)},
    \label{b4dimreg}
\end{split}
\end{equation}
where $M^2=M^2(s,m)=m^2-sx(1-x)$ and $s$ is the centre-of-mass energy, $s=p^2$.

Next, we can Wick rotate to Euclidean momenta and generalise this integral to $D$-dimensions. This is captured by the following identity, 
\begin{equation}
    \int\frac{d^Dq}{(2\pi)^D} \frac{q^{2a}}{(q^2-M^2)^b}=\frac{i}{(4\pi)^{D/2}}\frac{(-1)^{a-b}}{(M^2)^{b-a-D/2}} \frac{\Gamma(a+\frac{D}{2})\Gamma(b-a-\frac{D}{2})}{\Gamma(b)\Gamma(\frac{D}{2})}
\end{equation}
which is proven in the Appendix\footnote{if it exists.}. Here, $\Gamma(x)$ is the gamma function, where $\Gamma(x) = (x-1)! = (x-1)\Gamma(x-1)$, with $\Gamma(1)=1$. Moreover, for $D\neq 4$, the coupling $g$ for a $\phi^4$ theory will be dimensionful. We can keep $g$ dimensionless by performing the replacement
\begin{equation}
    g \rightarrow \mu^{4-D}g.
\end{equation}
Identifying $a=0$ and $b=2$ in Equation \ref{b4dimreg}, we have 

\begin{equation}
    \begin{split}
        I(p) &= ig\mu^{4-D} \frac{i}{(4\pi)^{D/2}}\frac{\Gamma(\frac{D}{2})\Gamma(2-\frac{D}{2})}{\Gamma(2)\Gamma(\frac{D}{2})}\int^1_0 dx \, \frac{(-1)^{-2}}{(M^2)^{2-D/2}},\\
        &= ig\mu^{4-D} \frac{i}{(4\pi)^{D/2}}\Gamma(2-\frac{D}{2})\int^1_0 dx \, \frac{1}{(M^2)^{2-D/2}},\\
    \end{split}
\end{equation}

Now, we let $D=4-2\epsilon$. Then, $I(p)$ becomes

\begin{equation}
    I(p)=ig\frac{i}{16\pi^2}(4\pi)^{\epsilon}\Gamma(\epsilon)\int^1_0 dx \, \left(\frac{\mu^2}{M^2}\right)^{\epsilon},
\end{equation}

Recall that, in the limit of $\epsilon\rightarrow 0$, we can make the following approximation:

\begin{equation}
\begin{split}
    u^\epsilon &= 1 + \epsilon \log{u} + \mathcal{O}(\epsilon),\\
    \Gamma(\epsilon) &= \frac{1}{\epsilon} - \gamma_E + \mathcal{O}(\epsilon),
\end{split}
\end{equation}
where $u$ is some polynomial of order $\epsilon$ and $\gamma_E$ is the Euler-Mascheroni constant.

Using the relations above, $I(p)$ becomes

\begin{equation}
    \begin{split}
       I(p)&=ig\frac{i}{16\pi^2}\int^1_0 dx \, (1+\epsilon\log{4\pi} + \mathcal{O}(\epsilon))(\frac{1}{\epsilon} - \gamma_E + \mathcal{O}(\epsilon))(1+\epsilon\log{\frac{\mu^2}{M^2}}+ \mathcal{O}(\epsilon)),\\
       &=ig\frac{i}{16\pi^2}\int^1_0 dx \, \frac{1}{\epsilon} - \gamma_E + \log{\frac{4\pi\mu^2}{M^2}} + \mathcal{O}(\epsilon).
    \end{split}
\end{equation}

Now, we apply the MS subtraction scheme, where the $\epsilon$ terms are removed from the integral. At last, Equation \ref{loopOg} becomes 

\begin{equation}
     \frac{i^2}{(p_1^2-m^2)(p_2^2-m^2)}\frac{-g}{16\pi^2}\int^1_0 dx \, \log{\frac{4\pi\mu^2}{M^2}} - \gamma_E.
\end{equation}
And this is the order 1 contribution to the correction from the first momentum region.

\subsection{Momentum region 2: large $q \sim O(1/x)$}
The remaining contribution is in equation (10.1.13):
\begin{equation}
    \frac{i^2}{(p_1^2-m^2)(p_2^2-m^2)}\frac{ig}{(2\pi)^4}\cross\qty{\int d^4q \frac{e^{iq\cdot x}-1}{(q^2-m^2)[(q-p_1-p^2)^2-m^2]}-\text{UV divergence}}
\end{equation}
For this term, the second region of momentum, where $q$ becomes large up to $O(1/x)$ as $x \rightarrow 0$, is important, since it provides a contribution of order 1 (as $q\cdot x \sim 1$), whereas the first region (finite $q$) only has a contribution of order $\abs{x}$. We again note that, as will be shown, the UV divergence here is equal to that present in the first region, and therefore the decomposition makes sense. 

As we are considering large $q \sim 1/x$, we can see that $p_1$, $p_2$ and $m$ does not provide leading-power contribution to the term in the curly-bracket (differentiating the term w.r.t $p_1$, $p_2$ or $m$ will always bring another power of $q$ to the denominator, which results in a convergent finite integral, which goes to $0$ as some power of $x$ when $x \rightarrow 0$). We can therefore ignore these variables and set $m=p_1=p_2=0$ in defining $c_1$. \\

Consider the term
\begin{equation}
\begin{split}
    &\ \text{i}\int d^4q\frac{e^{iq\cdot x}-1}{(q^2)^2} \\
    &\rightarrow\frac{\text{i}}{(2\pi\mu)^{D-4}}\int d^Dq\frac{e^{iq\cdot x}-1}{(q^2)^2} \text{ - analytically continue to dim D, with } g \rightarrow \mu^{4-D}g\\
    &= \frac{\text{ig}}{(2\pi\mu)^{D-4}}\int d^Dq \int_0^\infty  dz\ z e^{-z(-q^2)} e^{iq\cdot x}-1) \text{ - using identity provided}\\
    &= \frac{-1g}{(2\pi\mu)^{D-4}}\int d^Dq \int_0^\infty  dz\ z e^{-z(q^2)} (e^{-iq\cdot x}-1) \text{ - wick rotation}\\
    &= \frac{-1}{(2\pi\mu)^{D-4}}\int d^Dq \int_0^\infty  dz\ z (e^{-z(q^2+iq\cdot x/z+(ix/2z)^2)-x^2/4z}-e^{-z(q^2)}) \text{ - complete the square}\\
    &= \frac{-1}{(2\pi\mu)^{D-4}}\int d^Dq \int_0^\infty  dz\ z (e^{-z(q+ix/2z)^2}e^{-x^2/4z}-e^{-z(q^2)})\\
    &= \frac{-1}{(2\pi\mu)^{D-4}}\int_0^\infty  dz\ z (\frac{\pi}{z})^{D/2}(e^{-x^2/4z}-1)\text{ - gaussian integral}\\
\end{split}
\end{equation}

substitute $t = \frac{x^2}{4z},\ dz=-\frac{x^2}{4t^2}dt,\ \int_0^\infty \rightarrow \int_\infty^0$
\begin{equation}
\begin{split}    
    &= \frac{-1}{(2\pi\mu)^{D-4}}\int_0^\infty  dt\ \frac{x^2}{4t^2} \frac{x^2}{4t}
    (\frac{4\pi t}{x^2})^{D/2}(e^{-t}-1)\\
    &= \frac{-\pi^{D/2}(x^2)^{2-D/2}}{(2\pi\mu)^{D-4}4^{2-D/2}}\int_0^\infty  dt\ t^{D/2-3}(e^{-t}-1)
\end{split}
\end{equation}
using $\Gamma(z)=\int_0^\infty dt\ t^{z-1}e^{-t}$
\begin{equation}
    = \frac{-\pi^{D/2}(x^2)^{2-D/2}}{(2\pi\mu)^{D-4}4^{2-D/2}}(\Gamma(\frac{D}{2}-2)-\eval{\frac{t^{D/2-2}}{D/2-2}}_0^\infty )
\end{equation}
using the property that $\int d^Dp(p^2)^\alpha = 0$ (see Collins, p.73 eq 4.3.1a), we see that the second term vanishes. Now, we set $D=4-2\epsilon$
\begin{equation}
\begin{split}
    &= \frac{-\pi^{2-\epsilon}(x^2)^{\epsilon}}{(2\pi\mu)^{-2\epsilon}2^{2\epsilon}}\Gamma(-\epsilon)\\
    &= -\pi^2(\pi\mu^2 x^2)^\epsilon\Gamma(-\epsilon)\\
    &= -\pi^2 (1+\epsilon \ln(\pi \mu^2 x^2)+O(\epsilon^2))(-\frac{1}{\epsilon}-\gamma+O(\epsilon))\\
    &= \pi^2(\frac{1}{\epsilon}+\gamma+\ln(\pi\mu^2 x^2)+O(\epsilon))\\
    &= \pi^2(\frac{1}{\epsilon}+\gamma+\ln(-\pi\mu^2 x^2)+O(\epsilon))\text{ - Wick rotate back}
\end{split}
\end{equation}

Therefore
\begin{equation}
    \frac{ig}{(2\pi)^4}\cross\qty{\int d^4q \frac{e^{iq\cdot x}-1}{(q^2-m^2)[(q-p_1-p^2)^2-m^2]}} = \frac{1}{16\pi^2}\qty{\frac{1}{\epsilon}+\gamma+\ln(-\pi\mu^2 x^2)}
\end{equation}
As we can see, this term has the same pole/UV-divergence as that from the first region, and so the decomposition makes sense. And now we can identify
\begin{equation}
\begin{split}
    c_1(x) &= \frac{1}{2\pi^2} \qty{\frac{\text{i}}{(2\pi\mu)^{D-4}}\int d^Dq\frac{e^{iq\cdot x}-1}{(q^2)^2}-\frac{2}{D-4}}\\
    &=\frac{1}{2\pi^2} \qty{\pi^2(\frac{1}{\epsilon}+\gamma+\ln(\pi\mu^2 x^2)+O(\epsilon))-\frac{1}{\epsilon}}\\
    &=\frac{1}{2}(\gamma+\ln(-\pi\mu^2 x^2))
\end{split}
\end{equation}

\newpage

\bibliographystyle{ieeetr} 
\bibliography{opeBib.bib}

\end{document}
