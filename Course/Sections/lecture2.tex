\newcommand{\tphi}{\tilde{\phi}}
\newcommand{\tj}{\tilde{J}}
\newcommand{\tchi}{\tilde{\chi}}

\section{Free field theory}
\label{sec:real-scalar-field}

The lagrangian for a free real scalar field $\phi$ is given by
\begin{equation}
  \label{eq:RealScalLagr}
  \mathcal{L}_0\left(\phi(x)\right)=
  \frac12 \partial_\mu \phi(x) \partial^\mu \phi(x) - 
  \frac12 m^2 \phi(x)^2\, .
\end{equation}
The Euler-Lagrange equations of motion, 
\begin{equation}
  \label{eq:EulerLagrange}
  \partial_\mu \frac{\partial \mathcal{L}_0}{\partial\left(\partial_\mu \phi(x)\right)} 
  - \frac{\partial \mathcal{L}_0}{\partial \phi(x)}\, ,
\end{equation}
yields the Klein-Gordon equation
\begin{equation}
  \label{eq:KleinGordonScal}
  \partial_\mu \partial^\mu \phi(x) + m^2 \phi(x) = \left( \partial^2 + m^2 \right) \phi(x) = 0\, . 
\end{equation}

\paragraph{Lorentz Invariance}

Note that $\mathcal{L}_0$ is invariant under Lorentz transformations:
\begin{equation}
  \label{eq:LorentzTransfScal}
  x^\mu \mapsto x'^\mu =  \Lambda^\mu_\nu x^\nu\, , \quad \quad 
  \phi(x) \mapsto \phi'(x) = \phi(\lambda^{-1}x)\, .
\end{equation}

\paragraph{Conjugate Momentum} 

The momentum conjugate to $\phi(x)$ can be readily computed:
\begin{align}
  \label{eq:ScalMom}
  \Pi(x) &= \frac{\partial \mathcal{L}_0}{\partial (\partial_0\phi(x))} \\
  &= \partial_0 \phi(x) = \dot{\phi}(x)\, .
\end{align}
And therefore the Hamiltonian is 
\begin{align}
  \mathcal{H} &= \Pi(x) \dot{\phi}(x) - \mathcal{L}_0\left(\phi(x)\right) \\
  &= \frac12 \Pi(x)^2 + \frac12 \sum_{k=1}^3 \left(\partial_k \phi(x)\right)^2
    + \frac12 m^2 \phi(x)^2\, .
\end{align}
The quadratic Hamiltonian is the generalization of the harmonic oscillator to the case where we have an infinite number of canonical coordinates, indexed by the continuous spatial coordinate $\bfx$. 

\section{Path Integral }
\label{sec:path-integral}

The vacuum-to-vacuum amplitude in the presence of a source field $J(x)$ is the straightforward generalization of the expression we have derived for the quantum mechanical system. The correspondence between the two systems is as follows:

\bigskip

\begin{center}
  \begin{tabular}[!h]{lcl}
    $q(t)$ & $\longrightarrow$ & $\phi(t,\bfx)$ \\
    $\hat{Q}(t)$ & $\longrightarrow$ & $\hat{\phi}(t,\bfx)$ (operator) \\
    $f(t)$ & $\longrightarrow$ & $J(t,\bfx)$ (source) \\
  \end{tabular}
\end{center}

\bigskip

The projection onto the ground state is implemented by the $\epsilon$
trick that we introduced in the case of quantum mechanics, \ie by
replacing the Hamiltonian with $(1-i\epsilon) H$. In the case of the
scalar field theory, this is conveniently achieved by the substitution
$m^2 \mapsto m^2 - i\epsilon$. In all subsequent formulae we will
assume that the squared mass has an infinitesimal (negative) imaginary
part.   

By analogy with the QM computation, we can write the expression for
the path integral representation of the vacuum amplitude 
\begin{align}
  \label{eq:ScalPathIntDef}
  Z_0[J] = \braket{0}{0}_J = 
  \int \mathcal{D}\phi\, \exp\bigg\{
  i \Big[
  S_0[\phi] + J\cdot \phi
  \Big]
  \bigg\}\, ,
\end{align}
where 
\begin{align}
  S_0[\phi] = \int d^Dx\, \mathcal{L}_0\left(\phi(x)\right)\, ,
  \quad
  J \cdot \phi = \int d^Dx\, J(x) \phi(x)\, .
\end{align}
For a free theory the integral in \Eqref{eq:ScalPathIntDef} is a
Gaussian integral, similar to the ones we have seen in the first
lecture. In order to make the correspondence more explicit, we can
write the action as
\begin{align}
  S_0[\phi] = \int d^Dx d^Dx'\, \phi(x) K(x,x') \phi(x') + 
  \int d^Dx\, J(x) \phi(x)\, ,
\end{align}
where we recognise a quadratic term, with $K(x,x')$ playing the role
of $A_{ij}$, and $J(x)$ playing the role of the lienar term $b_i$. The
explicit expression for the kernel $K$ is
\begin{align}
  K(x,x') = \Big[ -\partial^2 - m^2\Big] \delta(x-x')\, .
\end{align}

It is convenient to work in momentum space, where the kernel
in the action is diagonal. Introducing the Fourier transforms
\begin{equation}
  \label{eq:ScalFieldFourier}
  \phi(x) = \int \frac{d^Dp}{(2\pi)^D}\, e^{-i p\cdot x}\,
  \tilde{\phi}(p)\, ,
\end{equation}
we can rewrite the kinetic term: 
\begin{align}
  \int d^Dx\, \partial_\mu \phi(x) \partial^\mu \phi(x) 
  &= \int d^Dx\, \int_{p,p'}\, (-ip_\mu) e^{-ip \cdot x} \tilde{\phi}(p) 
  (-ip'^\mu) e^{-ip'\cdot x} \tilde{\phi}(p') \\
  &= \int_p \, p^2 \tilde{\phi}(p) \tilde{\phi}(-p)
    =  \int_p p^2 \left|\tilde{\phi}(p)\right|^2\, ,
\end{align}
where we have introduced the notation
\begin{align}
  \int_p  = \int \frac{d^Dp}{(2\pi)^D}\, ,
\end{align}
and used the fact that $\phi(x)$ is real, and hence $\tilde{\phi}(-p) =
\tilde{\phi}(p)^*$. Hence the action for the free field in momentum
space can be written as
\begin{align}
  \label{eq:ScalFreeMomSpace}
  S_0[\phi] = \frac12 \int_p \bigg\{
  \tphi(-p) \left[p^2 - m^2 + i \epsilon\right] \tphi(p) +
  \tj(p) \tphi(-p) + \tj(-p) \tphi(p)
  \bigg\}\, .
\end{align}
Note that the contribution from the $\epsilon$ term to the exponential
is 
\begin{align}
  \exp\bigg\{
  -\epsilon \int_p \left|\tphi(p)\right|^2
  \bigg\}\, ,
\end{align}
which clearly is convergent for large values of $|\tphi(p)|$.
It is also important to note that the action for the free field is
quadratic, and the kernel is diagonal in momentum space. Therefore the
path integral for the free field is a simple extension of the Gaussian
integrals that we have been discussing in previous lectures. We will
use these previous results extensively. 

The Gaussian integral can be computed by performing the {\em usual}
shift of the integration variables
\[
  \tilde{\chi}(p) = \tphi(p) + \frac{\tj(p)}{p^2-m^2+i\epsilon}\, ,
\]
so that
\begin{align}
  S_0[\chi] = 
  \frac12 \int_p \bigg\{
  \tchi(-p) \left[p^2-m^2+i\epsilon\right] \tchi(p)
  + \tj(-p) \frac{1}{p^2-m^2+i\epsilon} \tj(p)
  \bigg\} \,.
\end{align}
Up to a normalization factor
\begin{align}
  Z_0[J] &\propto \exp \frac{i}{2} \int_p \tj(-p)
           \frac{1}{p^2-m^2+i\epsilon} \tj(p) \\
  &= \exp \frac{i}{2} \int d^Dx d^Dx'\,  J(x) \Delta(x,x') J(x') \, ,
\end{align}
where
\begin{align}
  \Delta(x,x') = \Delta(x-x') = 
  \int_p e^{-i p\cdot (x-x')} \frac{1}{p^2-m^2+i\epsilon}\, .
\end{align}
You can easily verify that $\Delta=K^{-1}$, \ie
\begin{align}
  \int d^Dz \, K(x,z) \Delta(z,x') = \delta(x-x')\, .
\end{align}
$\Delta$ is called the {\em Feynman propagator}.

Following our previous derivations for Gaussian integrals and QM, you
can show that
 \begin{align}
   \bra{0} T \phi(x_1) \phi(x_2) \ket{0}_0 &= 
   \left(\frac{1}{i} \frac{\delta}{\delta J(x_1)}\right)
   \left(\frac{1}{i} \frac{\delta}{\delta J(x_2)}\right)
   Z_0[J]\Big|_{J=0} \\ 
   &= \frac{1}{i} \Delta(x_1-x_2)\, .
 \end{align}
Further correlators are obtained by taking further more derivatives
\begin{align}
  \bra{0} T \phi(x_1) \ldots \phi(x_n) \ket{0}_0 
  &= 
    \left(\frac{1}{i} \frac{\delta}{\delta J(x_1)}\right)
    \ldots
    \left(\frac{1}{i} \frac{\delta}{\delta J(x_n)}\right)
    Z_0[J]\Big|_{J=0} \, .
\end{align}
They can be computed in the free theory using Wick's theorem, again
following the arguments we used for Gaussian integrals: 
\begin{align}
  \bra{0} T \phi(x_1) \phi(x_2) \phi(x_3) 
  &\phi(x_4) \ket{0}_0 
    = \frac{1}{i^2} \left[
    \Delta(x_1-x_2) \Delta(x_3-x_4) + \right. \nonumber \\
  & \left. + \Delta(x_1-x_2) \Delta(x_3-x_4) + 
    \Delta(x_1-x_2) \Delta(x_3-x_4)
    \right]\, .
\end{align}

\section{Interacting theory}
\label{sec:interacting-theory}

Let us now add an interaction term in the lagrangian:
\begin{align}
  \mathcal{L}\left(\phi(x)\right) = 
  \mathcal{L}_0\left(\phi(x)\right) + V\left(\phi(x)\right)\, .
\end{align}
We are going to consider several examples, \eg
\begin{align}
  V\left(\phi(x)\right) = \frac{1}{3!} g \phi(x)^3\, .
\end{align}
What is the dimension of the coupling $g$ as a function of $D$?

Denoting by $S_0[\phi]$ the action of the free theory, we can write
the path integral for the interacting theory
\begin{align}
  Z[J] = \braket{0}{0}_J = \int \mathcal{D}\phi\, 
  \exp\Bigg\{i \left( S_0[\phi] + \int d^Dx\,  V\left(\phi(x)\right) 
  + J \cdot \phi \right) \Bigg\}\, .
\end{align}
By performing the same manipulations that we discusssed for 
gaussian integrals we obtain
\begin{align}
  Z[J] &= \exp \Bigg\{i \int d^Dx\,  
         V\left(\frac{1}{i}\frac{\delta}{\delta J(x)}\right) \Bigg\}\, 
         \int \mathcal{D}\phi\, 
         \exp\Bigg\{i \left( S_0[\phi]  
         + J \cdot \phi \right) \Bigg\} \\
       &= \exp \Bigg\{i \int d^Dx\,  
         V\left(\frac{1}{i}\frac{\delta}{\delta J(x)}\right) \Bigg\}\, 
         Z_0[J]\, .
\end{align}
A useful expression is obtained by expanding both exponentials
\begin{align}
  Z[J] \propto 
  \sum_{V=0}^\infty & \frac{1}{V!} \left[
                      \frac{i g}{3!} \int d^Dx\, 
                      \left(\frac{1}{i}\frac{\delta}{\delta J(x)}\right)^3
                      \right]^V \times \\
  \label{eq:DoubleExpExp}
                    & \sum_{P=0}^\infty \frac{1}{P!} \left[
                      \frac{i}{2} \int d^Dy\, d^Dz\, 
                      J(y) \Delta(y-z) J(z)
                      \right]^P
\end{align}
Consider now the contribution for fixed values of $P$ and $V$, we are
left with $E=2P-3V$ sources. Let us look at the details that enter in
this contribution. 
\begin{itemize}
\item the overall factor of '$i$': 
  \begin{align}
    i^V \left(\frac{1}{i}\right)^{3V} i^P = i^{P-2V} = i^{E-P+V}\, .
  \end{align}
\item derivatives acting on sources: 
  \begin{align}
    \frac{(2P)!}{E!}\ \mathrm{combinations}\, .
  \end{align}
\end{itemize}
Many contractions yield the same result, which we will represent again
using a diagrammatic representation. 
\begin{itemize}
\item Propagators, $\Delta(x-y)$, are
  represented by a line connecting the points $x$ and $y$. 
  \begin{equation}
    \label{eq:DeltaXYFeynDiag}
    \frac{1}{i}\Delta(x-y) = 
    \begin{tikzpicture}[baseline={([yshift=1.4ex]current bounding box.center)}]
      \begin{feynman}[inline=(a)]
        \vertex (a);
        \vertex (b);
        \diagram {
          a -- b,
        };
        \vertex [below=0.2em of a] {\(_{x}\)};  
        \vertex [below=0.2em of b] {\(_{y}\)};  
      \end{feynman}
    \end{tikzpicture}\, .
  \end{equation}
  External sources are represented with a solid dot at the end of a
  line. Note that the solid dot includes the integration over $x$. 
  \begin{equation}
    \label{eq:CurrentXFeynDiag}
    i \int d^Dx\, J(x) = 
    \begin{tikzpicture}[baseline={([yshift=0.2ex]current bounding box.center)}]
      \begin{feynman}[inline=(a)]
        \vertex (a);
        \vertex (b);
        \diagram {
          a [dot] -- b,
        };
        \vertex [below=0.7em of a] {\(_{x}\)};  
        % \vertex [below=0.2em of b] {\(y\)};  
      \end{feynman}
    \end{tikzpicture}\, .
  \end{equation}
\item Interactions are represented as three-prong vertices, again
  including the integration over x.
  \begin{equation}
    \label{eq:VertXFeynDiag}
    i g \int d^Dx\,  = 
    \begin{tikzpicture}[baseline={([yshift=0.2ex]current bounding box.center)}]
      \begin{feynman}[small, inline=(v)]
        \vertex (v);
        \vertex [above=of v](i1);
        \vertex [below right=of v](i2); 
        \vertex [below left=of v](i3); 
        \diagram {
          (i1) -- (v),
          (v) -- (i2),
          (v) -- (i3),
        };
      \end{feynman}
    \end{tikzpicture}\, .
  \end{equation}  
\end{itemize}
We can then count the number of contractions that yield a particular
diagram, \ie. a particular contribution. Assuming that there is no
symmetry in the structure of the diagram, we have the following
possibilities. 
\begin{itemize}
\item permutations of the functional derivatives: $(3!)^V$;
\item permutations of the vertices: $V!$;
\item permutations of the sources at the end of propagators: $2^P$;
\item permutations of propagators: $P!$.
\end{itemize}
These factors match {\em exactly} the ones that appear in the
expansions of the exponentials above. So in the absence of any
symmetry in the diagram, each diagram contributes to
Eq.~(\ref{eq:DoubleExpExp}) multiplied by a factor of 1. However, this
procedure results in a double counting of the possible contributions
if there are symmetries in the structure of the diagram, \ie if a
permutation of derivatives results in the same operations as a
permutation of the sources. 

\bigskip

\noindent
{\bf Example:} let us consider the case $P=2$, $V=1$, and hence
$E=4-3=1$. The term in the double expansion is
\begin{align}
  \frac{i g}{3!} \int d^Dx\, 
  & \left(\frac{1}{i}\frac{\delta}{\delta J(x)}\right)
    \left(\frac{1}{i}\frac{\delta}{\delta J(x)}\right)
    \left(\frac{1}{i}\frac{\delta}{\delta J(x)}\right)\,
    \frac{1}{2!} \frac{i}{2}\, 
    \int d^Dy_1\, d^Dz_1\, J(y_1) \Delta(y_1-z_1) J(z_1) \nonumber \\  
  & \times \int d^Dy_2\, d^Dz_2\, J(y_2) \Delta(y_2-z_2) J(z_2)\, .
\end{align}
It can be rewritten as 
\begin{align}
  g \frac{1}{3! \times 2! \times 2 \times 2}\, 
  & \int d^Dx\, d^Dy_1\, d^Dz_1\, d^Dy_2\, d^Dz_2\, 
    \Delta(y_1-z_1) \Delta(y_2-z_2) \times \nonumber \\
  & \times \left(\frac{\delta}{\delta J(x)}\right)
    \left(\frac{\delta}{\delta J(x)}\right)
    \left(\frac{\delta}{\delta J(x)}\right)\,
    J(y_1) J(z_1) J(y_2) J(z_2)\, .
\end{align}
We are going to count the possible contractions in two different ways. 
\begin{enumerate}
\item Just counting: there are 4 possible choice to decide which $J$
  is not paired with a derivative. Then there are $3!$ possible ways
  of pairing the derivatives with the three $J$s. The derivatives
  acting on the $J$s produce Dirac deltas, which we use to evaluate
  some of the integrals. The final result is
  \begin{align}
    g & \frac{1}{3! \times 2! \times 2 \times 2}\, 4 \times 3! 
    \int d^Dx\, d^Dw\,  J(w) \Delta(w-x) \Delta(x-x) = \nonumber \\
    \label{eq:IntWithSymFactOne}
      & = g \frac12 \int d^Dx\, d^Dw\, J(w) \Delta(w-x) \Delta(0)\, .
  \end{align}
The integral can be represented diagrammatically as
\begin{equation}
  \begin{tikzpicture}[baseline={([yshift=0.2ex]current bounding box.center)}]
    \begin{feynman}
      \vertex (a);
      \vertex (b);
      \diagram {
        a [dot] -- b -- [out=45, in=-45, loop, min distance=2cm] b,
      };
      \vertex [below=0.7em of a] {\(_{w}\)};  
      \vertex [below=0.2em of b] {\(_{x}\)};  
    \end{feynman}
  \end{tikzpicture}
\end{equation}
Note that there is one source left in the integral, corresponding to
$E=1$, and hence one external 'dot' in the diagram. 
\item Let us now try to figure out the symmetry factor by working out
  the double counting in the general argument spelled out above. In
  that counting, swapping derivatives and swapping the ends of a
  propagator have been counted as distinct operations. However you can
  check that in this particular example swapping \eg $J(y_1)$ and
  $J(z_1)$ is equivalent to swapping the two derivatives that act on
  the currents. Hence, in order to take into account this double
  counting, we need to divide the integral that is
  represented by the diagram by a symmetry factor $S=2$, which
  precisely the factor of $1/2$ that appears in front of the integral
  in Eq.~(\ref{eq:IntWithSymFactOne}).
\end{enumerate}

\bigskip

\noindent {\bf Another example:} let us now take $P=3$ and $V=2$. The
number of external legs is $E=2\times 3 - 3 \times 2=0$. The overall
factor of i is $i^{P-2V}=-i$. The terms appearing in the expansion
are: 
\begin{align}
  \frac{1}{2!} 
  & g^2\, \frac{1}{(3!)^2} \frac{1}{2^3}\, 
    \int d^Dx_1\, d^Dx_2\, d^Dy_1\, d^Dz_1\, d^Dy_2\, d^Dz_2\,
    d^Dy_3\, d^Dz_3\, %\nonumber \\
  \Delta(y_1-z_1) \Delta(y_2-z_2) \Delta(y_3-z_3) \times
  \nonumber \\
  & \times \left(\frac{\delta}{\delta J(x_1)}\right)^3
    \left(\frac{\delta}{\delta J(x_2)}\right)^3
    J(y_1) J(z_1) J(y_2) J(z_2) J(y_3) J(z_3) \, .
\end{align}
Let us focus on the contractions that yield the following diagrammatic
representation
\begin{equation}
  \begin{tikzpicture}[baseline={([yshift=0.2ex]current bounding box.center)}]
    \begin{feynman}
      \vertex (a);
      \vertex (b);
      \diagram {
        a -- [out=120, in=-120, loop, min distance=2cm] a -- b -- [out=60, in=-60, loop, min distance=2cm] b,
      };
      \vertex [below=0.2em of a] {\(_{x}\)};  
      \vertex [below=0.2em of b] {\(_{x'}\)};  
    \end{feynman}
  \end{tikzpicture} 
\end{equation} 
and let us try to work out the
combinatorial factor in two ways. 
\begin{enumerate}
\item Just counting: we need to pair each of the sources in the
  integrand with one of the derivatives. Starting from $y_1$ we have
  2 ways of choosing which group of derivatives to use, then 3 choices
  for picking one derivative. The source in $z_1$ then needs to couple
  with the one the remaining derivatives in the same group, so there
  are 2 choices. The three remaining sources need then to be paired
  with the three remaining derivatives, for which there are $3\times
  2$ choices. Hence we obtain
  \begin{align}
    g^2 \frac{1}{2^3} \int d^Dx d^Dx' \Delta(0)^2 \Delta(x-x')\, .
  \end{align}
\item Symmetry factor: swapping the ends of the propagator that
  appears as a loop closing at $x$ is equivalent to swapping two
  derivatives at $x$, which yields a factor of 2; swapping the ends of
  the propagator that appears as a loop closing at $x'$ is equivalent
  to swapping two derivatives at $x'$, which yields another factor of
  2; and finally swapping the ends of the propagator from $x$ to $x'$
  is equivalent to swapping the two groups of three derivatives, which
  yields yet another factor of 2. Hence in total we have $S=2^3$,
  which is consistent with the result above. 
\end{enumerate}

The only way to get these things right is by practicing -- a lot. There are
plenty of examples in Srednicki's book! We should try them together,
maybe in a tutorial. 

\section{Disconnected diagrams}
\label{sec:disc-diagr}

A disconnected diagram is made of the product of several connected
pieces. We denote by $C_I$ the contribution of a given connected
region, including its symmetry factor, and $D$ the total contribution
of the diagram. Then
\begin{equation}
  \label{eq:TotDiscDiag}
  D = \frac{1}{S_D} \prod_I \left(C_I\right)^{n_I}\, ,
\end{equation}
where $n_I$ is the number of times that the subdiagram $C_I$ appears.
In order to evaluate properly the contribution of the total
disconnected diagram we need to evaluate $S_D$, \ie find out the
possible double counting that is left after the symmetry factors of
each connected subdiagram has been worked out. The only residual
symmetry that is left in the total diagram comes from the exchange of
{\em all} propagators and vertices amongst different but identical
connected subdiagrams. The symmetry factor can be readily evaluated
\begin{equation}
  \label{eq:OverallSymFact}
  S_D = \prod_I n_I!\, .
\end{equation}
We now have all the ingredients to write the sum of diagrams that
contribute to the generating functional:
\begin{align}
  \label{eq:GenFuncSum}
  Z[J] &= \sum_{\{n_I\}} D(\{n_I\}) 
  \propto \sum_{\{n_I\}} \prod_I \frac{1}{n_I!} \left(C_I\right)^{n_I}
  \\
       &= \prod_I \left\{ \sum_{n_I=0}^\infty \frac{1}{n_I!}
         \left(C_I\right)^{n_I}\right\} \\
       &= \exp\left\{\sum_I C_I\right\}\, ,
\end{align}
which shows that $\log Z[j]$ is the sum of all connected diagrams.

We can now address the issue of the normalization of $Z[J]$. We can
request that $Z[0]=1$, or equivalently define
\begin{equation}
  \label{eq:NormGenFunc}
  \frac{Z[J]}{Z[0]} = \exp\left\{ 
    i W[J]
    \right\}\, ,
\end{equation}
where 
\begin{equation}
  \label{eq:ConnectGenerator}
  i W[J] = \sum_I{\vphantom{\sum}}' C_I\, , 
\end{equation}
and the prime in the sum indicates that we do not include the vacuum
diagrams in the sum, \ie the diagrams with $E=0$.
