\newcommand{\Hin}{\mathcal{H}_{\mathrm{in}}}
\newcommand{\matH}{\mathcal{H}}
\newcommand{\Dspace}{\scriptscriptstyle{D-1}}
\newcommand{\Dall}{\scriptscriptstyle{D}}
\newcommand{\instate}{\mathrm{in}}
\newcommand{\outstate}{\mathrm{out}}
\newcommand{\dddt}{\frac{\overleftrightarrow{\partial}}{\partial t}}

\section{Time Evolution Pictures in Quantum Mechanics}
\label{sec:reps-quant}

In order to set the notation, we briefly summarise the different representations
used to describe the time evolution of quantum systems. The state of the system
at $t=0$ is described by a vector $\ket{\psi} \in \mathcal{H}$, where
$\mathcal{H}$ is the Hilbert space of physical states.

\paragraph{Schr\"odinger Picture}

In the Schr\"odinger representation the time evolution is encoded in the time
dependence of the state vector, $\ket{\psi_S(t)}$, which evolves according to
the Schr\"odinger equation
\begin{equation}
    \label{eq:SchrodEq}
    i \partial_t \ket{\psi_S(t)} = H \ket{\psi_S(t)}\, ,
\end{equation}
where $H$ is the Hamiltonian operator for the system under study. The state of
the system at time $t$ is 
\begin{equation}
    \label{eq:SchrTimeEvol}
    \ket{\psi_S(t)} = e^{-i H t} \ket{\psi}\, .
\end{equation}
Operators
associated to observables, $O_S$, are time independent (unless the observable
itself has an explicit dependence on time). The expectation value of the
observable at time $t$ is 
\begin{equation}
    \label{eq:SchrodExp}
    O(t) = \langle \psi_S(t) | O_S | \psi_S(t) \rangle\, .
\end{equation}

\paragraph{Heisenberg Picture}

In the Heisenberg representation, the state vector does not evolve in time. At
all times $t$, $\ket{\psi_H(t)}=\ket{\psi}$. We will therefore drop the suffix
$H$ when we refer to states in the Heisenberg representation. Time evolution is
encoded in the time dependence of the operators $O_H(t)$. Clearly the
expectation value at time $t$ should not depend on whether we work in the
Schr\"odinger or Heisenberg representation:
\begin{equation}
    \label{eq:HeisenExp}
    O(t) = \langle \psi | O_H(t) | \psi \rangle\, .
\end{equation}
Comparing Eqs.~\eqref{eq:SchrodExp} and~\eqref{eq:HeisenExp} we deduce that
\begin{equation}
    \label{eq:HeisenOp}
    O_H(t) = e^{iHt} O_S e^{-iHt}\, ,
\end{equation}
and therefore 
\begin{equation}
    \label{eq:HeisenEvol}
    i \frac{d}{dt} O_H(t) = - \left[H, O_H(t)\right]\, ,
\end{equation}
where $H$ is the Hamiltonian again.~\footnote{Clearly, the Hamiltonian operator
in the Heisenberg representation is time-independent. Do not confuse the
Hamiltonian $H$ and the suffix $H$, used to denote the Heisenberg
representation.}

\paragraph{Interaction Picture}

The interaction picture interpolates between the previous two. In this case the
Hamiltonian is divided into a {\it free} Hamiltonian $H_0$ and an interaction
term $V$. The time dependent state vector in the interaction picture is defined
by `subtracting' the free evolution from the Schr\"odinger state vector at time
$t$:
\begin{equation}
    \label{eq:IntPictEvol}
    \ket{\psi_{\mathrm{int}}(t)} = e^{i H_0 t} \ket{\psi_S(t)}\, .
\end{equation}
The operators are also time-dependent:
\begin{equation}
    \label{eq:IntPictOps}
    O_{\mathrm{int}}(t) = e^{i H_0 t} O_S e^{-i H_0 t}\, .
\end{equation}
In this representation the free Hamiltonian is time independent,
\begin{equation}
    \label{eq:H0IntPict}
    H_{0,\mathrm{int}}(t) = H_0\, ,
\end{equation}
and the time evolution of the states is described by a first-order differential
equation similar to Schr\"odinger's equation, where $V_{\mathrm{int}}$ replaces $H$,
\begin{equation}
    \label{eq:IntPictEvolEq}
    i \partial_t \ket{\psi_{\mathrm{int}}(t)} = V_{\mathrm{int}}(t) \,
    \ket{\psi_{\mathrm{int}}(t)}\, .
\end{equation}

\paragraph{Fields} Note that fields in QFT are treated as operator-valued 
distributions and therefore
\begin{equation}
    \label{eq:HvIntFields}
    \phi_H(t,\mathbf{x}) = U(t)^\dagger \phi_{\mathrm{int}}(t,\mathbf{x}) 
    U(t)\, ,
\end{equation}
where
\begin{equation}
    \label{eq:Uoperator}
    U(t) = e^{iH_0t} e^{-iHt}\, .
\end{equation}

\paragraph{Conventions for Generators of Translations}

Using a $D$-dimensional covariant notation, and a Minkovski metric 
\begin{equation}
    \label{eq:MostlyMinus}
    \eta_{\mu\nu} = \mathrm{diag}\left\{1, -1, -1, \ldots \right\}\, ,
\end{equation}
we represent the $D$-dimensional momentum operator in position space as
\begin{equation}
    \label{eq:MomOp}
    P^\mu = i \partial^\mu\, ,
\end{equation}
which reduces to the usual expressions for the Hamiltonian $P^0$ and the spatial
components of the momentum $P^k$. The momentum being the generator of
translations, the wave function of a system in the Schr\"odinger picture obeys
\begin{equation}
    \label{eq:PsiTranslation}
    \psi\left(x+a\right) = e^{-i P\cdot a} \psi\left(x\right)\, ,
\end{equation}
while for the field operators in the Heisenberg representation
\begin{equation}
    \label{eq:PhiTranslation}
    \phi_H(x+a) = e^{i P\cdot a} \phi_H(x) e^{-i P\cdot a}\, .
\end{equation}

\section{Asymptotic States}
\label{sec:AsymptStates}

`In' and `Out' states are states of the interacting theory in the Heisenberg
representation, which are characterized by the behaviour of the system in the
far past and the far future respectively, where the particles can be considered
to be well separated. The separation between particles implies that these states
should be in a one-to-one correspondence with free-particle states. Let us focus
here on in-states, similar results hold for the out-states. The Hilbert space of
in-states is denoted $\Hin$. A basis of $\Hin$ is made of simultaneous momentum
eigenstates of $N$ particles, where $N$ spans the set of integer numbers
\begin{equation}
    \label{eq:HinBasis}
    \left\{ \ket{k_1 \ldots k_N; \mathrm{in}}, N \in \mathbb{N} \right\}\, .
\end{equation}
Clearly the individual momenta are not conserved quantities, while translation
invariance guarantees that the total momentum $P^\mu$ is conserved,
\begin{equation}
    \label{eq:TotMomConserv}
    P^\mu \ket{k_1 \ldots k_N; \mathrm{in}} = 
    \left(\sum_{m=1}^N k^\mu_m\right) 
    \ket{k_1 \ldots k_N; \mathrm{in}}\, .
\end{equation}
From the definitions summarised in Sec.~\ref{sec:reps-quant}, we see that states
in the Schr\"odinger and Heisenberg pictures coincide at $t=0$. We can express
the fact that in-states behave like free-particle states as $T \to -\infty$. Let
$\ket{a; 0}$ be a free-particle state, we require
\begin{equation}
    \label{eq:InAsympOne}
    \lim_{T\to-\infty} e^{-i H_0 T} \ket{a;0} =
    \lim_{T\to-\infty} e^{-i H T} \ket{a; \mathrm{in}}\, ,
\end{equation}
which we rewrite as~\footnote{There is potential for sloppiness here, but I
don't think it does matter.}
\begin{align}
    \label{eq:InAsympTwo}
    \ket{a; \mathrm{in}} &= \lim_{T\to-\infty} e^{i H T} e^{-i H_0 T} \ket{a;0}
    \\
    \label{eq:MollerOne}
    &= \Omega^+ \ket{a; 0}\, .
\end{align}
Similarly for out-states,
\begin{align}
    \label{eq:OutAsympTwo}
    \ket{a; \mathrm{out}} &= \lim_{T\to +\infty} e^{i H T} e^{-i H_0 T} \ket{a;0}
    \\
    \label{eq:MollerTwo}
    &= \Omega^- \ket{a; 0}\, .
\end{align}
Finally we recall the definition of the $S$ matrix: 
\begin{align}
    \label{eq:SMatOne}
    S_{ab} &= \langle a; \mathrm{out} | b; \mathrm{in} \rangle \\
    \label{eq:SMatTwo}
    &= \langle a; 0 | \left(\Omega^-\right)^\dagger
    \Omega^+ | b; 0\rangle\, .
\end{align}

\section{Axiomatic Field Theory - States and Fields}
\label{sec:AFTONe}

We work with fields in the Heisenberg picture, denoted $\phi(x)$. The axioms
needed to build a quanutm theory of fields can be divided into three main
categories: axioms that specify the states of the theory, axioms that establish
the properties of fields, and axioms that define the field-particle duality. In
this section we look at the first two categories, while deferring the discussion
of field-particle duality to a later section. 

\paragraph{State Axioms.}

The structure of the Hilbert space of quantum states, $\mathcal{H}$, is defined
by the following axioms. 

\begin{enumerate}
    \item [{\bf Ia.}] $\mathcal{H}$ is a separable~\footnote{
        A separable space contains a countable, dense subset. For now, we are not 
        going to pay too much attention to the places where this property is used. 
    } Hilbert space, which carries
    a unitary representation $U(\Lambda,a)$ of the Poincar\'e group, 
    \begin{equation}
        \label{eq:PoincaRep}
        \ket{\alpha} \in \matH\, , \quad U(\Lambda,a) \ket{\alpha} \in \matH\, .
    \end{equation}
    
    \item [{\bf Ib.}] The spectrum of $P^\mu$ lies in the forward light-cone,
    \begin{equation}
        \label{eq:ForwSpec}
        P^0 \geq 0\, , \quad P^2 \geq m^2\, ,
    \end{equation}
    where $m$ is the mass of the single-particle state. 

    \item [{\bf Ic.}] There is a unique vacuum state $\ket{0}$, normalized such that $\langle 0 | 0\rangle = 1$. Note that this assumption implies the absence of spontaneous symmetry breaking. 
    
    \item [{\bf Id.}] Existence of a mass gap, the spectrum of $P^2$ is empty between 0 and $m^2$.
\end{enumerate}

\paragraph{Field Axioms.}

This set of axioms specifies the properties of quantum fields and their
expectation values. Details of the theory of distributions are summarised 
in Chapter~\ref{chap:distr-notes}.

\begin{enumerate}
    \item [{\bf IIa.}] The Heisenberg fields $\phi(x)$ are operator-valued distributions, \ie
    \begin{equation}
        \label{eq:OpValuedDistr}
        \forall f \in \mathcal{S}(\mathbb{R}^D)\, , 
        \phi_f = \int d^Dx\, f(x) \phi(x)
    \end{equation}
    is an unbounded operator on $D \subset \matH$,~\footnote{
        Here we use $D$ to denote a dense set, {\em not} the number of 
        spacetime dimensions. Hopefully the meaning of the symbol is clear
        from the context. 
    } where $D$ is dense in $\matH$
    and $\mathcal{S}$ is the set of Schwartz test functions. The image of $D$ is
    a subset of $D$ itself, which allows to perform subsequent applications of
    the smeared operators, \eg
    \begin{align}
        \langle 0 | \phi_{f_1} \ldots \phi_{f_n} | 0\rangle 
        &= \int d^Dx_1 \ldots d^Dx_n\, f_1(x_1) \ldots f_n(x_n) \, 
        W(x_1 \ldots x_n) \, , \\
        W(x_1 \ldots x_n)
        &= \langle 0 | \phi(x_1) \ldots \phi(x_n) | 0\rangle\, .
    \end{align} 
    The correlators $W(x_1 \ldots x_n)$ are called {\em Wightman functions}. The
    Wightman functions are {\em tempered}\ distributions, \ie
    \begin{equation}
        \label{eq:TempDistr}
        W(z) = D^m F(z)\, , \quad F(z) < C (1 + |z_E|^2)^p\, ,
    \end{equation}
    for some integer values of $p$. Note that $z$ is a generalized coordinate
    that collects $x_1, \ldots x_n$, $m=(m_1 \ldots m_n)$ is a vector of
    integers, $D^m$ is a short-hand for 
    \begin{equation}
        \label{eq:MultiDeriv}
        D^m = 
        \frac{\partial^{|m|}}{\partial x_1^{m_1} \partial x_2^{m_2} \ldots}\, ,
    \end{equation}
    and $|m| = m_1 + \ldots + m_n$.

    \item [{\bf IIb.}] Under Poincar\'e transformations $\left\{\Lambda, a\right\}$
    \begin{align}
        & U(\Lambda,a)\, \phi_f\, U(\Lambda,a)^\dagger 
            = \phi_{f_{\Lambda,a}} \, , \\
        & f_{\Lambda,a}(x) = f\left(\Lambda^{-1}(x-a)\right)\, .
    \end{align}

    \item [{\bf IIc.}] If $f_1$ and $f_2$ have compact supports $V_1$ and $V_2$ respectively, and if 
    \[
        \forall x_1 \in V_1\, , \forall x_2 \in V_2 \, ,    
        \quad (x_1 - x_2)^2 < 0\, ,
    \]
    then 
    \begin{equation}
        \label{eq:SmearedLightConeComm}
        \left[ \phi_{f_1} , \phi_{f_2} \right] = 0\, .
    \end{equation}

    \item [{\bf IId.}] Polynomials of $\phi_f$, with all possible choices of $f$, 
    acting on $\ket{0}$ form a dense set in $\matH$.
\end{enumerate}

Because of translation invariance, the Wightman functions only depend on $n-1$
variables
\begin{equation}
    \label{eq:WightFunTransInv}
    W(x_1 \ldots x_n) = \mathcal{W}\left(\xi_1 \ldots \xi_{n-1}\right)\, ,
\end{equation}
where $\xi_i = x_i - x_{i+1}$. They can be extended by
analytical continuation to the tube $\mathcal{T}_n$, as discussed in Chapter~\ref{chap:distr-notes}, 
$\zeta_i = \xi_i - i \eta_i$, with $\eta_i^2>0$.
The latter functions are invariant under complex Lorentz transformations
\begin{equation}
    \label{eq:ComplexLorentzInv}
    \mathcal{W}(\Lambda\zeta_1 \ldots \Lambda\zeta_{n-1}) =
    \mathcal{W}(\zeta_1 \ldots \zeta_{n-1})\, ,
\end{equation}
which in turn implies that $\mathcal{W}$ is an analytical function of the scalar
products $\zeta_i \cdot \zeta_j$. Note that the Euclidean {\em Schwinger
functions}\ are defined by analytical continuation~\footnote{In Euclidean space 
the spacetime vectors $x_E$ have $D$ components that are related to the Minkowski 
ones as follows
\[
 x_E^i = x^i\, ,~\mathrm{for}~i=1, \ldots,D-1\, , \quad x_E^D = i x^0\, .   
\] }
\begin{equation}
    \label{eq:SchwingFunDef}
    S(x_1 \ldots x_n) = 
        W\left((-i x_1^D, \mathbf{x}_1) 
        \ldots (-i x_n^D, \mathbf{x}_n)\right)\, .
\end{equation}

We can define an $x$-dependent smeared field
\begin{align}
    \label{eq:XdepPhiSmeared}
    \phi_f(x) 
    &= e^{i P \cdot x} \phi_f e^{-i P \cdot x} \\
    &= \int d^Dz\, f(z-x) \phi(z)\, ,
\end{align}
and bilocal smeared operators
\begin{equation}
    \label{eq:BilocalSmeared}
    \phi_f = \int d^Dx_1 d^Dx_2\, f\left(x_1 - x_2\right) 
    \phi(x_1) \phi(x_2)\, .
\end{equation}
The Fourier transform of a local smeared operator
\begin{equation}
    \label{eq:FourierSmeared}
    \tilde\phi_f(p) = \int d^Dx\, e^{i p\cdot x} \phi_f(x)
\end{equation}
carries exactly momentum $p$. The smearing of the field in position space does
not affect the momentum of the Fourier transform, and hence
\begin{equation}
    \label{eq:DeltaMomSmeared}
    \langle \alpha | \tilde\phi_p(p) | \beta\rangle 
    \propto \delta\left(P_\alpha-P_\beta-p\right)\, .
\end{equation}

\subsection{An Example Proof: Space-like Commutation}
\label{sec:ExampleProof}

In order to develop some familiarity with the concepts we introduce above, it is
interesting to work out explicitly the proof of the following property. Let
$a=\left(0, \mathbf{a}\right)$, then axiom {\bf IIc} implies
\begin{equation}
    \label{eq:FasterThanPoly}
    \left| \langle 0 |
        \left[ \phi_{f_1}(-a), \phi_{f_2}(a)\right]
        | 0 \rangle 
    \right| < C \left|\mathbf{a}\right|^{-N}\, , \quad \forall N\, , 
    \mathrm{for}~\left|\mathbf{a}\right|\to\infty\, .
\end{equation}

\paragraph{Proof}

The expectation value in \eqref{eq:FasterThanPoly} can be written as
\begin{align}
    \label{eq:ExpComm}
    \langle 0 |
        \left[ \phi_{f_1}(-a), \phi_{f_2}(a)\right]
        | 0 \rangle =
        \int d^Dx_1 d^Dx_2\, f_1(x_1) f_2(x_2)\,
        \langle 0 | 
        \left[\phi(x_1-a), \phi(x_2+a)\right]
        | 0 \rangle\, .
\end{align}
Introducing $2D$-dimensional vectors $z=(x_1,x_2)$ and $\alpha=(a,-a)$, we define 
$F(z) = f_1(x_1) f_2(x_2)$, which is a fast-decreasing function of $z$, and
\begin{align}
    \label{eq:CommZvar}
    \langle 0 | 
        \left[\phi(x_1-a), \phi(x_2+a)\right]
        | 0 \rangle 
        &= W\left(x_1-a,x_2+a\right) - W\left(x_2+a,x_1-a\right) \\
        &= W\left(z-\alpha\right)\, .
\end{align} 
We can then rewrite
\begin{equation}
    \label{eq:SmearedCorrZvar}
    \langle 0 |
        \left[ \phi_{f_1}(-a), \phi_{f_2}(a)\right]
        | 0 \rangle 
        = \int dz\, F(z) W(z-\alpha)\, ,
\end{equation}
where $W(z)$ is a tempered distribution, 
\begin{align}
    \label{eq:WTempDist}
    W(z) &= D^m \mathcal{P}(z)\, , \\
    \mathcal{P}(z) &= c_1 \left(1 + |z|_E^2\right)^p\, , \quad \mathrm{for\ all\ } p\, .
\end{align}
If $|z|_E<|\mathbf{a}|$ then the separation between $(x_1-a)$ and $(x_2+a)$ is space-like 
and $W(z-\alpha)$ vanishes, hence, after integrating by parts, 
\begin{equation}
    \label{eq:CutIntegralW}
    \left|\langle 0 |
        \left[ \phi_{f_1}(-a), \phi_{f_2}(a)\right]
        | 0 \rangle \right| = \int_{|z|_E>|\mathbf{a}|}
        dz\, \mathcal{P}(z-\alpha)\, D^mF(z)\, .
\end{equation}
Because $W$ is a tempered distribution, we know that 
\begin{equation}
    \label{eq:PFastDecrease}
    \mathcal{P}(z-\alpha) < c_1 \left(1 + |z-\alpha|_E\right)^p 
    < c_1 \left(1 + |z|_E\right)^p  \left(1 + |\alpha|_E\right)^p \, ,
\end{equation}
for all values of $p$. The smearing functions are also fast decreasing, therefore for 
any value of $N$, 
\begin{equation}
    \label{eq:SmearFastDecrease}
    \left| D^m F(z)\right| < c_2 \left|z\right|_E^{-(N+2p+2D)}\, \left(1 + |z|_E^2\right)^p \, ,
\end{equation}
where $p$ is the same power that appears in \eqref{eq:PFastDecrease}. Finally, in spherical coordinates, 
we have that 
\[
    d^Dz = S_{2D}\, |z|_E^{2D-1} d|z|_E\, .   
\]
Putting everything together and performing the integral over $z$ outside the sphere of radius 
$|\mathbf{a}|$ yields
\begin{equation}
    \label{eq:BoundOnSmearCorr}
    \left|\langle 0 |
        \left[ \phi_{f_1}(-a), \phi_{f_2}(a)\right]
        | 0 \rangle \right| < 
        c_1 c_2 \frac{S_{2D}}{N+2p} \left(1 + 2 |\mathbf{a}|\right)^p
        |\mathbf{a}|^{-N-2p}\, , 
\end{equation}
which shows explicitly the fast decay of the correlator as a function of the distance $|\mathbf{a}|$.

\section{Ruelle Clustering Theorem}
\label{sec:RuelleClust}

Connected correlators of smeared fields can be defined in the usual way:
\begin{align}
    &\langle 0 | \phi_{f_1}(x_1) | 0 \rangle_c 
        = \langle 0 | \phi_{f_1}(x_1) | 0 \rangle \, , \\
    &\langle 0 | \phi_{f_1}(x_1) \phi_{f_2}(x_2) | 0 \rangle_c 
        = \langle 0 | \phi_{f_1}(x_1) \phi_{f_2}(x_2) | 0 \rangle -
           \langle 0 | \phi_{f_1}(x_1) | 0 \rangle \langle 0 | \phi_{f_2}(x_2) | 0 \rangle\, , \\
    &\ldots \nonumber
\end{align}
The Clustering Theorem states that: 

\begin{Thm}
    for a set of coordinates at equal time $x_i=(t,\mathbf{x}_i)$, 
    with $d=\max_{i,j} |\mathbf{x}_i - \mathbf{x}_j|$, for any $N>0$, 
    the {\em connected} correlators satisfy
    \begin{equation}
        \label{eq:RuelleClusteringThm}
        \langle 0 | \phi_{f_1}(x_1) \ldots \phi_{f_n}(x_n) | 0 \rangle_c < C_N\, d^{-N}\, .
    \end{equation}        
\end{Thm}

We are going to summarise the proof of the theorem for $n=2$ only; as we will see, the existence of a 
mass gap in the theory plays an important role in establishing the clustering property. Let us consider 
$a,b<m$, where $m$ is the mass gap of the theory. We introduce the function 
\begin{equation}
    \label{eq:FPlusFunction}
    F_+\left(p^0\right) = 
    \begin{cases}
        0\, &\quad \mathrm{for}\ p^0<a , \\
        \frac{\exp\left[-\frac{\kappa}{\left(p^0-a\right)^2}\right]}
        {\exp\left[-\frac{\kappa}{\left(p^0-a\right)^2}\right] + \exp\left[-\frac{\kappa}{\left(p^0-b\right)^2}\right]} 
        \, &\quad \mathrm{for}\ a<p^0<b , \\
        1\, &\quad \mathrm{for}\ b<p^0\, ,
    \end{cases}
\end{equation}
and then 
\begin{align}
    \label{eq:FMinusFunction}
    F_-\left(p^0\right) 
        &= F_+\left(-p^0\right)\, , \\
    \label{eq:FZeroFunction}
    F_0\left(p^0\right)
        &= 1 - F_+\left(p^0\right) - F_-\left(p^0\right)\, .
\end{align}
The three functions are shown in Fig.~\ref{fig:PartitionOfUnity}. 
Notice that $F_0$ has a compact support in a range below $m$, \ie
where the spectrum of the theory is empty. Similarly $F_-$ vanishes 
for all positive values of the energy.  

\begin{figure}[ht]
    \centering
    \includegraphics[scale=0.75]{Sections/plots/PartitionOfUnity.png}
    \caption{The three functions $F_+$, $F_-$ and $F_0$, for 
    $\kappa=0.1$, $a=1/4$, $b=3/4$, in units of $m$.}
    \label{fig:PartitionOfUnity}
\end{figure}

These functions can be used to decompose every smeared field 
\begin{equation}
    \label{eq:FieldDecompF}
    \phi_f(x) = \phi_f^{(+)}(x) + \phi_f^{(-)}(x) + \phi_f^{(0)}(x)\, ,
\end{equation}
where 
\begin{align}
    \phi_f^{(+)}(x) 
        &= \int \frac{d^Dp}{(2\pi)^D}\, e^{-ip\cdot x}\, F_+\left(p_0\right)\, 
        \tilde{\phi}_f\left(p\right) \\
        &= \int \frac{d^Dp}{(2\pi)^D}\, d^Dy\, e^{-ip\cdot (x-y)}\, F_+\left(p^0\right)\,
        \phi_f(y)\, .
\end{align}
Similar relations define $\phi_f^{(-)}$ and $\phi_f^{(0)}$. Note that each of
these three fields is a bona fide smeared field itself. 

Using the properties of the spectrum of $p^0$, we have
\begin{equation}
    \label{eq:FieldsAndSpectrum}
    \begin{cases}
        &\phi_f^{(-)}(x) |0\rangle = 0 \, , \\        
        &\langle 0 | \phi_f^{(+)}(x) = 0 \, , \\
        & \phi_f^{(0)}(x) |0\rangle = C |0\rangle\, .
    \end{cases}
\end{equation}
We can therefore decompose the correlator
\begin{equation}
    \label{eq:CorrelatorSplit}
    \begin{split}
        \langle 0 | &\phi_{f_1}(x_1) \phi_{f_2}(x_2) | 0\rangle = 
            \langle 0 | \phi_{f_1}^{(0)}(x_1) \phi_{f_2}^{(0)}(x_2) | 0\rangle + \\
        & + \langle 0 | \left[\phi_{f_1}^{(0)}(x_1), \phi_{f_2}^{(+)}(x_2)      \right] | 0\rangle 
        + \langle 0 | \left[\phi_{f_1}^{(-)}(x_1), \phi_{f_2}^{(+)}(x_2)\right] | 0\rangle \\
        & + \langle 0 | \left[\phi_{f_1}^{(-)}(x_1), \phi_{f_2}^{(0)}(x_2)  \right] | 0\rangle \, .
    \end{split}
\end{equation}
Now for the commutators that appear in Eq.~\eqref{eq:CorrelatorSplit}, we
already established that they vanish faster than any power of the distance
between the points. For the first term
\begin{align}
    \langle 0 | \phi_{f_1}^{(0)}(x_1) \phi_{f_2}^{(0)}(x_2) | 0\rangle 
    &= \langle 0 | \phi_{f_1}^{(0)}(x_1) | 0\rangle\, 
        \langle 0 | \phi_{f_2}^{(0)}(x_2) | 0\rangle \\
    &=  \langle 0 | \phi_{f_1}(x_1) | 0\rangle\, 
    \langle 0 | \phi_{f_2}(x_2) | 0\rangle \, .
\end{align}
Taking this term to the left-hand side yields the desired bound for the
connected correlator. 

\section{Axiomatic Field Theory - Particle-Field Duality}
\label{sec:PartFieldDual}

Finally we have two axioms that relate the fields and their correlators to the
physical states. 

\begin{enumerate}
    \item [{\bf IIIa.}] For some one-particle state
    \begin{equation}
        \label{eq:OnePartStateEx}
        \ket{\alpha} = \int \frac{d^{\Dspace}k}{(2\pi)^{\Dspace}}\, 
        g(\mathbf{k}) \, \ket{\mathbf{k}}\, , \quad g\left(\mathbf{k}\right) \in L_2\, ,
    \end{equation}
    corresponding to a discrete $m^2$ eigenvalue of $P^2$, the smeared field $\phi_f(x)$ has a 
    non-vanishing matrix element
    \begin{equation}
        \label{eq:PhiFME}
        \bra{0} \phi_f(x) \ket{\alpha} \neq 0\, .
    \end{equation}
    The field $\phi_f$ is called an interpolating Heisenberg field for the 
    state $\ket{\alpha}$.
    \item [{\bf IIIb.}] We denote by $\matH_\mathrm{in}$ the Hilbert space of
    states of far-separated, freely-moving stable particles in the far {\em
    past}. And similarly $\matH_\mathrm{out}$ for the same states in the far
    {\em future}. We postulate that  $\matH_\mathrm{in}$ and
    $\matH_\mathrm{out}$ are unitarily equivalent to $\matH$, which can be
    generated by applications of smeared fields. 
\end{enumerate}

\section{Haag-Ruelle Scattering Theory}
\label{sec:HaagRuelleScat}

We start by constructing a smeared field such that
\begin{enumerate}
    \item [(a)] acting on the vacuum, it produces single-particle states only; 
    \item [(b)] these states are time-independent.
\end{enumerate}
The construction proceeds in two steps. First we build a smeared field
$\phi_1(x)$ that satisfies property (a). The suffix `1' denotes precisely the
fact that this field overlpas with one-particle states. Then, by further
smearing, we ensure that the state is time-independent. 

\subsection{One-Particle States}
\label{sec:OnePartStat}

We start by considering a smearing function in momentum space
$\tilde{f}^{(1)}(p)$, which has its support for $am^2 < p^2 < bm^2$, where
$0<a<1$ and $1<b<4$. We choose $\tilde{f}^{(1)}(p)$ to be $C^\infty$ and
fast-decreasing. The Fourier transform of $\tilde{f}^{(1)}(p)$ is 
\begin{equation}
    \label{eq:FourierInvSmear}
    f^{(1)}(x) = \int \frac{d^{\Dall}\!p}{(2\pi)^{\Dall}}\, 
    e^{-ip\cdot x}\, \tilde{f}^{(1)}(p)\, .
\end{equation}
We use this smearing function to define the field 
\begin{equation}
    \label{eq:SmearedOnePart}
    \phi_1(x) = \int d^{\Dall}\!z\, f^{(1)}(z-x)\, \phi(z)\, .
\end{equation}
The matrix element of $\phi_1$ between the vacuum and a one-particle state with 
momentum $\mathbf{k}$ is 
\begin{equation}
    \label{eq:OnePartPhi}
    \bra{\mathbf{k}} \phi_1(x) \ket{0} = 
        e^{ik\cdot x} \tilde{f}^{(1)}(E_k,\mathbf{k})\, \bra{\mathbf{k}} \phi(0) \ket{0}\, ,
\end{equation}
where we used $E_k=\sqrt{\mathbf{k}^2+m^2}$.
By Axiom {\bf IIIa} the field $\phi_1$ will create a one-particle state when acting on 
the vacuum. Because of the properties of the smearing function $f^{(1)}$, it will {\em only}\ 
produce a one-particle state. The matrix element on the right-hand side of Eq.~\eqref{eq:OnePartPhi} 
is a scalar, 
\ie it is a number that is independent of $\mathbf{k}$, and is fixed by the normalization of 
the elementary field $\phi$ and the normalization of the states $\ket{\mathbf{k}}$. We choose to 
normalize the field $\phi$ in the interacting theory so that its matrix element with the one-particle state
is equal to the one in the free theory~\footnote{Here we are going to diverge from the conventions
used in Duncan's book and stick to the Lorentz-invariant normalization.}, \ie
\begin{equation}
    \label{eq:InteractFieldNorm}
    \bra{\mathbf{k}} \phi(0) \ket{0} = 1\, . 
\end{equation}

\paragraph{Digression on Normalizations}

The free scalar field is decomposed in a superposition of creation and annihilation 
operators
\begin{equation}
    \label{eq:FreeScalModeDecomp}
    \phi(x) = 
    \int \frac{d^{\Dspace} \!p}{(2\pi)^{\Dspace} 2E_p}\,
    \left\{
        e^{-ip\cdot x} a(\mathbf{p}) + e^{ip\cdot x} a(\mathbf{p})^\dagger
    \right\}\, ,    
\end{equation}
where the creation and annihilation operators have the {\em canonical} commutation 
relations
\begin{equation}
    \label{eq:CreatAnnihilComm}
    \left[a(\mathbf{p}), a(\mathbf{p}')\right] 
    = 2 E_p (2\pi)^{\Dspace} \delta\left(\mathbf{p}-\mathbf{p}'\right)\, .
\end{equation}
One-particle states are constructed acting with a creation operator on the vacuum
\begin{equation}
    \label{eq:OnePartCreation}
    \ket{\mathbf{k}} = 
    a(\mathbf{k})^\dagger \ket{0}\, ,
\end{equation}
and obey a Lorentz-invariant normalization condition
\begin{equation}
    \label{eq:OnePartLorentzNorm}
    \langle \mathbf{k} | \mathbf{k}' \rangle = 
    2E_k \, (2\pi)^{\Dspace} \delta\left(\mathbf{k}-\mathbf{k}'\right) \, .
\end{equation}
Using these conventions, the matrix element of the free field with the one-particle state is 
\begin{equation}
    \label{eq:FreeOnePartME}
    \bra{\mathbf{k}} \phi(x) \ket{0} = e^{ik\cdot x}\, ,
\end{equation}
where $k^0=E_k=\sqrt{\mathbf{k}^2 + m^2}$.

\subsection{Time-Independent States}
\label{sec:TimeIndepStates}

Let us consider a positive-energy solution of the Klein-Gordon equation
\begin{equation}
    \label{eq:KGSolution}
    g\left(t,\mathbf{x}\right) = 
        \int \frac{d^{\Dspace}\!p}{(2\pi)^{\Dspace}\, 2E_p}\,
        e^{-ip\cdot x}\, \tilde{g}(\mathbf{p})\, ,
\end{equation}
where $p\cdot x = E_p t - \mathbf{p}\cdot\mathbf{x}$ and $\tilde{g}(\mathbf{p})$ is 
$C^\infty$ and fast decreasing. The smeared operator
\begin{equation}
    \label{eq:Phi1gDef}
    \phi_{1g}(t) = -i 
        \int d^{\Dspace}\!x \left\{ g(t,\mathbf{x}) 
        \frac{\overleftrightarrow{\partial}}{\partial t} \phi_1(t,\mathbf{x})\right\} 
\end{equation}
is such that 
\begin{equation}
    \label{eq:TimeIndepOnePartState}
    \frac{d}{dt} \phi_{1g}(t) \ket{0} = 0\, .
\end{equation}
The proof is straightforward, using the fact that $g$ is a solution of Klein-Gordon, 
integration by parts and the fact that $\phi_1(x) \ket{0}$ is a one-particle state. 
It can be readily checked that the wave function of the state in momentum space is 
\begin{equation}
    \label{eq:MomentumWaveFuncState}
    \begin{split}
        \psi_{1g}(\mathbf{k}) 
            &= \bra{\mathbf{k}} \phi_{1g}(t) \ket{0} \\
            &= \tilde{g}(\mathbf{k})\, \tilde{f}^{(1)}(E_k,\mathbf{k})\, .
    \end{split}
\end{equation}

\subsection{Asymptotic Behaviour}
\label{sec:AsympTimeBehaviour}

An important ingredient in deriving Haag's Asymptotic Theorem is the 
behaviour of $g(t,\mathbf{x})$ for large values of $t$. Setting 
$\mathbf{x}=\mathbf{v} t$, we can rewrite
\begin{equation}
    \label{eq:LargeTExp}
    g\left(t,\mathbf{x}\right) = 
        \int \frac{d^{\Dspace}\!p}{(2\pi)^{\Dspace}\, 2E_p}\,
        e^{it \left(\mathbf{p}\cdot \mathbf{v} - E_p\right)}\, 
        \tilde{g}(\mathbf{p})\, .
\end{equation}
For large $t$ and fixed $\mathbf{v}$, we can evaluate the integral expanding around 
its stationary point,
\begin{equation}
    \label{eq:StationaryPoint}
    \frac{\partial}{\partial p_k} \left(\mathbf{p}\cdot \mathbf{v} - E_p\right) = 
    v_k - \frac{p_k}{E_p}=0 \quad \Longrightarrow \quad 
    \mathbf{p} = m \gamma \mathbf{v}\, , \quad 
    \gamma = \frac{1}{\sqrt{1-v^2}}\, .
\end{equation}
Expanding to second order around the stationary point yields
\begin{equation}
    \label{eq:StatPtExpand}
    \left(\mathbf{p}\cdot \mathbf{v} - E_p\right) = 
    -\frac{m}{\gamma} - \frac12 \left(p-m\gamma v\right)_i
    \mathcal{M}_{ij} \left(p-m\gamma v\right)_j\, ,
\end{equation}
where 
\begin{equation}
    \label{eq:MMatrixAbove}
    \mathcal{M}_{ij} = \frac{1}{m\gamma} \, 
    \left(\delta_{ij} - v_i v_j\right)\, .
\end{equation}
In $D-1$ spatial dimensions this matrix has eigenvalues $1/m\gamma^3$ and $1/m\gamma$, 
with degeneracies 1 and $D-2$ respectively, and hence
\begin{equation}
    \label{eq:DetMMatrixAbove}
    \det \mathcal{M} = \frac{1}{\gamma^2} \left(\frac{1}{m\gamma}\right)^{D-1}\, .
\end{equation}
Putting everything together, we get the asymptotic behaviour
\begin{equation}
    \label{eq:GFunAsymp}
    g\left(t,\mathbf{v}t\right) \sim
        C \left|t\right|^{-(D-1)/2}\, e^{-imt/\gamma}\, 
        \left[\gamma^{(D-1)/2} \tilde{g}\left(m\gamma \mathbf{v}\right) +
        O\left(1/t\right)\right]\, .
\end{equation}


\subsection{Two Lemmas}
\label{sec:TwoLemmas}

Let us now consider a smeared field~\footnote{Note that the field $\phi_{1,g}$ that we 
considered above is the sum of two such fields.} 
\begin{equation}
    \label{eq:GenericGSmearing}
    \phi_{1,g}(t) = \int d^{\Dspace}x\, g(t,\mathbf{x}) \phi_{1}(t,\mathbf{x})\, .
\end{equation}

\paragraph{Lemma 1}

For large times, $t\to\pm\infty$, correlators of the smeared field 
\begin{equation}
    \label{eq:SmearedCorrs}
    \mathcal{M}_{m,n}(t) = \langle 0 | \phi_{1,g'_1}(t)^\dagger \ldots \phi_{g'_m}(t)^\dagger \, 
    \phi_{1,g_1}(t) \ldots \phi_{1,g_n}(t) | 0 \rangle
\end{equation}
behave like
\begin{equation}
    \label{eq:LemmaOne}
    \begin{cases}
        & O\left(|t|^{-(D-1)/2}\right)\, , \quad \mathrm{for}\ m\neq n\, , \\
        & \\
        & \sum_\mathrm{pairs} \prod_p 
            \langle 0 | 
            \phi_{1,g'_{i_p}}(t)^\dagger
            \phi_{1,g_{j_p}}(t) 
            | 0\rangle + 
            O\left(|t|^{-(D-1)}\right)\, , \quad \mathrm{for}\ m=n \, .
    \end{cases}
\end{equation}

\noindent
{\bf Proof}\ 

\noindent
The correlator $\mathcal{M}_{m,n}$ has a cluster expansion, \ie it
can be written as a sum of terms in which the $m+n$ fields are distributed in
$N_c$ clusters made of connected correlators. We denote by $m_r$ and $n_r$ the
number of $\phi^\dagger$ and $\phi$ fields respectively that appear in the
$r$-th cluster, with $r=1, \ldots N_c$. The contribution of the $r$-th cluster
can be written as
\begin{align}
    \label{eq:RthClusterContrib}
    \int d^{\Dspace}x_1 \ldots &d^{\Dspace}x_{m_r+n_r}\, G_1(t,\mathbf{x}_1)
      \ldots G_{m_r+n_r}(t,\mathbf{x}_{m_r+n_r})\, \times \nonumber \\
        &\times \langle \phi_{1}(t,\mathbf{x}_1) \ldots 
            \phi_{1}(t,\mathbf{x}_{m_r+n_r})
        \rangle_c \, ,
\end{align}
where $G_i(t,\mathbf{x})$ denotes either $g(t,\mathbf{x})$ or
$g(t,\mathbf{x})^*$, and the field $\phi_{1}$ may also represent a
$\phi_{1}^\dagger$. We can rewrite the integral above by introducing the
relative distance from $\mathbf{x}_1$:
\begin{equation}
    \label{eq:ShiftedClusterIntegral}
    \begin{split}
        \int & d^{\Dspace}x_1 G_1(t,\mathbf{x}_1) \, \\
        &\int d^{\Dspace}\xi_2 
            \ldots d^{\Dspace}\xi_{m_r+n_r}\, 
            G_2(t,\mathbf{x}_1+\mathbf{\xi}_2) 
            \ldots G_{m_r+n_r}(t,\mathbf{x}_1+\mathbf{\xi}_{m_r+n_r}) 
        \\
        & \quad \times \langle\phi_{1}(t,0) \phi_{1}(t,\mathbf{\xi}_2) \ldots 
        \phi_{1}(t,\mathbf{\xi}_{m_r+n_r}) \rangle_c\, , 
    \end{split}
\end{equation}
where translation invariance was used in the last line. 

Let us now investigate the asymptotic behaviour
of these contributions for large times $t$. From Ruelle's Clustering Theorem, we
know that the connected correlator in Eq.~\eqref{eq:ShiftedClusterIntegral} is a
fast decreasing function of the space vectors $\mathbf{\xi}_i$. Setting
$\mathbf{x}_1=\mathbf{v}_1 t$, we have
\begin{equation}
    \label{eq:AsymptoticBehaviourGn}
    \left|G_n(t,t\mathbf{v}_1+\mathbf{\xi}_n)\right| \sim
    |t|^{-(\Dspace)/2}\, \gamma_1^{(\Dspace)/2} \, 
    \left|\tilde{G}_n(m\gamma_1 \mathbf{v}_1)\right|\, , \quad \mathrm{for}\ t\to\pm\infty\, .
\end{equation}
where, as usual, $\gamma_1=(1-\mathbf{v}_1^2)^{-1/2}$. Finally, we note that 
\begin{equation}
    \int d^{\Dspace}x_1 \longrightarrow 
    t^{\Dspace}\, \int_{|\mathbf{v}_1|<1} d^{\Dspace}v_1 \,.
\end{equation}
Let us remember that we are considering the contribution where the fields are
distributed amongst $N_c$ clusters, and therefore
\begin{equation}
    \label{eq:CountingFieldsInClusters}
    \sum_{r=1}^{N_c} m_r = m\, , \quad 
    \sum_{r=1}^{N_c} n_r = n\, , \quad 
    m+n = N\, .
\end{equation}
Collecting all the terms that we have discussed above we see that
\begin{equation}
    \label{eq:ClusterAsymptTotal}
    \mathcal{M}_{m,n}(t) \sim t^{(\Dspace) \left(N_c - N/2\right)}\, .
\end{equation}
Since $\phi_{1}$ only overlaps with one-particle states, 
\begin{equation}
    \label{eq:OneParticlePhase}
    \langle 0 | \phi_{1}(0) | 0 \rangle = 0\, .
\end{equation}
Therefore every cluster must have at least {\bf two} fields. If each cluster has
exactly two fields, then $N_c-N/2=0$ and the contribution of that particular
configuration does not vanish as a power of $t$. On the other hand, if at least
one cluster contains three fields, then $N \geq 2(N_c-1) + 3$, which in turn
implies
\begin{equation}
    \label{eq:ThreeFieldClusterSuppression}
    N_c - N/2 \leq -\frac12\, ,
\end{equation}
and therefore those contributions are suppressed by powers of $t$. So the only
contributions that survive in the large-time limit are the ones where each
cluster contains exactly one pair of fields, as stated by the Lemma. 

\paragraph{Lemma 2 (Permutation Symmetry)}

Consider the two states
\begin{align}
    \label{eq:ProdPhiState}
    \ket{\Psi(t)} 
        &= \phi_{1,g_{1}}(t) \ldots \phi_{1,g_{n}}(t) \ket{0}\, , \\
    \label{eq:ProdPhiStatePerm}
    \ket{\Psi'(t)}
        &= \phi_{1,g_{P_1}}(t) \ldots \phi_{1,g_{P_n}}(t) \ket{0}\, , 
\end{align}
where $\left\{P_1 \ldots P_n\right\}$ is a permutation of $\left\{1 \ldots
n\right\}$, then 
\begin{equation}
    \label{eq:PermutationSymmetry}
    || \left(\ket{\Psi'(t)} - \ket{\Psi(t)}\right) || \sim
    t^{-(\Dspace)/2}\, .
\end{equation}
We will not discuss the proof of this Lemma in this notes. 

We end this section by noting that these two Lemmas guarantee that fields that
have been smeared with solutions of the Klein-Gordon equation behave in the
large-time limit like elementary fields in the free theory. Lemma 1 is
reminescent of Wick's theorem, while Lemma 2 reflects the usual symmetry under
permutations that we have in a Fock space constructed with
creation and annihilation operators. 

\subsection{Haag's Asymptotic Theorem}
\label{sec:HaggAsympThm}

The theorem guarantees
that states built from smeared interacting fields converge for asymptotic times to 
the in and out states of the theory. 

\paragraph{Theorem}

The state
\begin{equation}
    \label{eq:HaagAsymptState}
    \ket{\Psi(t)} = 
    \phi_{1,g_1}(t) \ldots \phi_{1,g_n}(t) \ket{0}
\end{equation}
converges strongly for $t \to -\infty$ to the $n$-particle {\em in-state}
\begin{align}
    \ket{\Psi_\mathrm{in}} 
        &= \ket{g_1 \ldots g_n; \instate} \\
        &= \int d^{\Dspace}k_1 \ldots d^{\Dspace}k_n\,
            \psi_{1,g_1}(\mathbf{k}_1) \ldots \psi_{n,g_n}(\mathbf{k}_n) \,
            \ket{\mathbf{k}_1 \ldots \mathbf{k}_n; \instate}\, ,
\end{align}
where the wave functions $\psi_{1,g_i}(\mathbf{k}_i)$ have been introduced in 
\eqref{eq:MomentumWaveFuncState},
\begin{equation}
    \psi_{1,g_i}(\mathbf{k}_i) =
    \bra{\mathbf{k}_i} \phi_{1,g_i}(t) \ket{0}\, .
\end{equation}

\noindent
{\bf Sketch of the proof}

\noindent
Using the previous results we can show that the time derivative of the state
$\ket{\Psi(t)}$ is $O(t^{-(\Dspace)/2})$. Therefore 
\begin{equation}
    \label{eq:CauchySeries}
    || \left(\ket{\Psi(t)} - \ket{\Psi(t')}\right) || 
        < \frac{C}{T^{-(D-3)/2}}\, , 
    \quad \mathrm{for}\ t,t'>T\, .
\end{equation}
Hence the sequence of states is a Cauchy series and does converge to a given 
state, which we identify with $\ket{\Psi_\mathrm{in}}$. The smeared fields define 
wave packets whose overlap vanishes as we go to asymptotic times. 

\subsection{Asymptotic Condition}
\label{sec:AsymptCondition}

We can use the Haag-Ruelle results to derive the connection between the Heisenberg 
field $\phi(x)$ and the free `in' field $\phi_\mathrm{in}$. 

\paragraph{Smeared Heisenberg Field}

The starting point of 
our analysis is a smeared field $\phi_g(t)$, defined as the $\phi_{1,g}$ fields above
except for the fact that the smearing function $f$ is a generic Schwartz function, 
whose support in momentum space does not need to sandwich the one-particle mass 
hyperboloid. The function $g(t,\mathbf{x})$ is still a positive-energy solution of 
the Klein-Gordon equation. Under these hypotheses, $\phi_{g} \ket{0}$ is no longer 
time-independent, nor is it a one-particle state, but Lemma 1 can still be used and 
therefore
\begin{align}
    \bra{0} &\phi_{1,g'_1}(t)^\dagger \ldots \phi_{1,g'_m}(t)^\dagger \, 
        \phi_{g}(t) \, 
        \phi_{1,g_1}(t) \ldots \phi_{1,g_n}(t) \ket{0} = \nonumber \\
        &= \sum_{i=1}^m \bra{0} \phi_{1,g'_i}(t)^\dagger \phi_{g}(t) \ket{0}_c \,
        \bra{0} \phi_{1,g'_1}(t)^\dagger \ldots \widehat{\phi_{1,g'_i}(t)^\dagger}
        \ldots \phi_{1,g'_m}(t)^\dagger \, 
        \phi_{1,g_1}(t) \ldots \phi_{1,g_n}(t) \ket{0} \nonumber \\
        &+ \sum_{i=1}^n \bra{0} \phi_{g}(t) \phi_{1,g_i}(t) \ket{0}_c \,
        \bra{0} \phi_{1,g'_1}(t)^\dagger \ldots \phi_{1,g'_m}(t)^\dagger \, 
        \phi_{1,g_1}(t) \ldots \widehat{\phi_{1,g_i}(t)}
        \ldots \phi_{1,g_n}(t) \ket{0} \nonumber \\
        \label{eq:HaagRuelleOne}
        & + O\left(|t|^{-(\Dspace)/2}\right)\, ,
\end{align}
where the `hat' indicates that the field drops from the correlation function. Let us
now assume that for some reason, possibly a symmetry of the theory, 
\begin{equation}
    \label{eq:SymmetryCondition}
    \bra{0} \phi_{g}(t) \ket{0} = 0 \, ,
\end{equation}
so that we can replace the connected two-point correlator in \eqref{eq:HaagRuelleOne}
with the ordinary two-point one: 
\begin{align}
    \label{eq:ConnectNormalReplaceOne}
    \bra{0} \phi_{1,g'_i}(t)^\dagger \phi_{g}(t) \ket{0}_c &= 
    \bra{0} \phi_{1,g'_i}(t)^\dagger \phi_{g}(t) \ket{0} \\
    \label{eq:ConnectNormalReplaceTwo}
    \bra{0} \phi_{g}(t) \phi_{1,g_i}(t) \ket{0}_c &= 
    \bra{0} \phi_{g}(t)  \phi_{1,g_i}(t) \ket{0} \, .
\end{align}
The field $\phi_{1,g}$ only overlaps with one-particle states, according to 
Haag's theorem
\begin{align}
    \label{eq:OneParticleOverlap}
    \phi_{1,g_j}(t) \ket{0} 
        &= \ket{g_j; \mathrm{in}} = \int \frac{d^{\Dspace}k}{(2\pi)^{\Dspace} 2E_k}\, 
        \psi_{1,g_j}(\mathbf{k}) \ket{\mathbf{k};\instate} \, , \\
    \bra{0} \phi_{1,g'_i}(t)^\dagger
        &= \bra{g'_i;\instate} = \int \frac{d^{\Dspace}k}{(2\pi)^{\Dspace} 2E_k}\, 
        \psi_{1,g'_i}(\mathbf{k})^* \bra{\mathbf{k};\instate} \, .
\end{align}
Finally, for a smeared field $\phi_{f}$,
\begin{equation}
    \label{eq:SmearedFieldOneParticleME}
    \bra{\mathbf{k};\instate} \phi_f(x) \ket{0} = 
        Z^{1/2} \tilde{f}(\mathbf{k}) e^{ik\cdot x}\, .
\end{equation}
Here and below, keep in mind that the momenta are on-shell, 
\begin{equation}
    \label{eq:OnShellMomenta}
    k\cdot x = E_k t - \mathbf{k} \cdot \mathbf{x}\, .
\end{equation}
Collecting all these equations, we deduce
\begin{align}
    &\bra{0} \phi_{1,g'_i}(t)^\dagger \phi_{g}(t) \ket{0} 
        = \int \frac{d^{\Dspace}k}{(2\pi)^{\Dspace} 2E_k}\, 
        \psi_{1,g'_i}(\mathbf{k})^* \bra{\mathbf{k};\instate} \phi_{g}(t) \ket{0} \\
    & \quad = \int \frac{d^{\Dspace}k}{(2\pi)^{\Dspace} 2E_k}\, d^{\Dspace}x\, 
        \psi_{1,g'_i}(\mathbf{k})^* \left\{ -i g(x) 
        \frac{\overleftrightarrow{\partial}}{\partial t}
        \bra{\mathbf{k};\instate} \phi_{f}(t) \ket{0} \right\} \\
    & \quad = -i \int \frac{d^{\Dspace}k}{(2\pi)^{\Dspace} 2E_k}\, d^{\Dspace}x\, 
        \frac{d^{\Dspace}p}{(2\pi)^{\Dspace} 2E_p}\,
        \psi_{1,g'_i}(\mathbf{k})^* \tilde{g}(\mathbf{p}) 
        \left\{ e^{-ip\cdot x}
        \frac{\overleftrightarrow{\partial}}{\partial t}
        e^{i k\cdot x}\right\} \tilde{f}(\mathbf{k})\, .
\end{align}
This expression can be simplified using the following relations: 
\begin{align}
    \label{eq:OrthoRelationOne}
    \int d^{\Dspace}x\, 
        \left\{ e^{-ip\cdot x}
            \frac{\overleftrightarrow{\partial}}{\partial t}
            e^{i k\cdot x}
        \right\} &= 
        i\, 2E_k (2\pi)^{\Dspace} \delta\left(\mathbf{p}-\mathbf{k}\right)\, , \\
    \label{eq:OrthoRelationTwo}
    \int d^{\Dspace}x\, 
        \left\{ e^{-ip\cdot x}
            \frac{\overleftrightarrow{\partial}}{\partial t}
            e^{-i k\cdot x}
        \right\} &= 
        0\, ,    
\end{align}
Using Eq.~\eqref{eq:OrthoRelationOne} yields
\begin{align}
    \label{eq:PhigMatrixElement}
    \bra{0} \phi_{1,g'_i}(t)^\dagger \phi_{g}(t) \ket{0} 
    &= Z^{1/2} \int \frac{d^{\Dspace}k}{(2\pi)^{\Dspace} 2E_k}\,
    \psi_{1,g'_i}(\mathbf{k})^* \psi_g(\mathbf{k})\, , \\
    \psi_g(\mathbf{k})
    &= \tilde{g}(\mathbf{k})\, \tilde{f}(\mathbf{k})\, .
\end{align}
The second relation, Eq.~\eqref{eq:OrthoRelationTwo}, guarantees that 
\begin{equation}
    \bra{0} \phi_{g}(t)  \phi_{1,g_i}(t) \ket{0} = 0\, .
\end{equation}
The bottom line of this lengthy computation is 
\begin{align}
    &\bra{g'_1 \ldots g'_m; \instate} \phi_{g}(t) \ket{g_1 \ldots g_n; \instate} 
    \nonumber \\
    \label{eq:SmearedFieldInstateME}
    & \quad \stackrel{t\to\infty}{\longrightarrow}
    Z^{1/2} \sum_{i=1}^m \int \frac{d^{\Dspace}k}{(2\pi)^{\Dspace} 2E_k}\,
    \psi_{1,g'_i}(\mathbf{k})^* \psi_g(\mathbf{k})\,
    \braket{g'_1 \ldots \widehat{g'_i} \ldots g'_m; \instate} 
    {g_1 \ldots g_n; \instate}\, .
\end{align}

\paragraph{Free Asymptotic Field}

The matrix element in Eq.~\eqref{eq:SmearedFieldInstateME} can be compared with the
matrix element of the smeared field $\phi_{\instate,g}$, which is defined using the same 
smearing procedure, but starting from a free local field 
\begin{equation}
    \label{eq:InFieldDefinition}
    \phi_\instate(x) =
    \int \frac{d^{\Dspace} \!p}{(2\pi)^{\Dspace} 2E_p}\,
    \left\{
        e^{-ip\cdot x} a_\instate(\mathbf{p}) + 
        e^{ip\cdot x} a_\instate(\mathbf{p})^\dagger
    \right\}\, ,
\end{equation}
where $a_\instate$ and $a_\instate^\dagger$ are the annihilation and creation 
operators in $\mathcal{H}_\instate$:
\begin{align}
    a_\instate(\mathbf{k}) \ket{\mathbf{k}_1 \ldots \mathbf{k}_n; \instate} 
    &= \sum_{r=1}^n 2E_k (2\pi)^{\Dspace} \delta\left(\mathbf{k}-\mathbf{k}_r\right) 
    \ket{\mathbf{k}_1 \ldots \widehat{\mathbf{k}}_r \ldots \mathbf{k}_n; \instate}\, , \\
    a_\instate(\mathbf{k})^\dagger \ket{\mathbf{k}_1 \ldots \mathbf{k}_n; \instate} 
    &= \ket{\mathbf{k} \mathbf{k}_1 \ldots \mathbf{k}_n; \instate} \, .
\end{align}
The contribution from the annihilation operator to the matrix element vanishes 
because of Eq.~\eqref{eq:OrthoRelationTwo} again, and hence we only need to compute 
the matrix element of the creation part of the smeared field 
\begin{equation}
    \label{eq:CreationInSmeared}
    \phi_{\instate,g}(t) = \int \frac{d^{\Dspace} k}{(2\pi)^{\Dspace} 2 E_k}\, 
    \psi_{g}(\mathbf{k}) a_\instate(\mathbf{k})^\dagger\, ,
\end{equation}
which yields 
\begin{align}
    \bra{g'_1 \ldots g'_m; \instate} &\phi_{\instate,g} 
    \ket{g_1 \ldots g_n;\instate} = \nonumber \\
    \label{eq:InFieldInstateME}
    &=\sum_{i=1}^m \int \frac{d^{\Dspace} k}{(2\pi)^{\Dspace} {2 E_k}}\, 
    \psi_{1,g'_i}(\mathbf{k})^* \psi_{g}(\mathbf{k})
    \braket{g'_1 \ldots \widehat{g'_i} \ldots g'_m; \instate}{g_1 \ldots g_n;\instate}\, .
\end{align}
Comparing Eqs.~\eqref{eq:SmearedFieldInstateME} and~\eqref{eq:InFieldInstateME}, 
we see that the two matrix elements coincide in the limit $t\to -\infty$, up to a 
normalization factor $Z^{1/2}$. Because the sets of states run over a dense subset of 
$\mathcal{H}_\instate$, we can deduce that the smeared field $\phi_{g}(t)$ converges to 
$\phi_{\instate,g}$ in a {\em weak}\ sense, thus establishing the {\em asymptotic condition}
\begin{equation}
    \label{eq:AsymptConnection}
    \bra{\beta;\instate} \phi_{g}(t) \ket{\alpha;\instate} 
    \stackrel{t\to-\infty}{\longrightarrow} 
    Z^{1/2} \bra{\beta;\instate} \phi_{\instate,g}(t) \ket{\alpha;\instate} \, .
\end{equation}

\section{LSZ Reduction}
\label{sec:LSZRedux}

The asymptotic condition is the starting point to derive the LSZ reduction formula. Noting that
the creation part of the smeared in-field $\phi_{\instate,g}(t)$ is time-independent, we can 
think of it as a smeared creation operator of an in-state, 
\begin{equation}
    \label{eq:SmearedInCreation}
    a_{\instate,g}^\dagger = \int \frac{d^{\Dspace} k}{(2\pi)^{\Dspace} 2 E_k}\, 
    \psi_{g}(\mathbf{k}) a_\instate(\mathbf{k})^\dagger\, .
\end{equation}
Using asymptotic completeness, i.e. the fact that $\mathcal{H}_\instate$ and $\mathcal{H}_\outstate$ 
coincide, we can write the asymptotic condition at early times as
\begin{align}
    -i \int d^{\Dspace}x \, g(x) \dddt \bra{\beta; \outstate} \phi(x) \ket{\alpha;\instate}
    \stackrel{t\to -\infty}{\longrightarrow} 
    Z^{1/2} \bra{\beta; \outstate} a_{\instate,g}^\dagger \ket{\alpha;\instate} \, ,\\
    i \int d^{\Dspace}x \, g(x)^* \dddt \bra{\beta; \outstate} \phi(x) \ket{\alpha;\instate}
    \stackrel{t\to -\infty}{\longrightarrow} 
    Z^{1/2} \bra{\beta; \outstate} a_{\instate,g} \ket{\alpha;\instate} \, ,
\end{align}
while for large times 
\begin{align}
    -i \int d^{\Dspace}x \, g(x) \dddt \bra{\beta; \outstate} \phi(x) \ket{\alpha;\instate}
    \stackrel{t\to +\infty}{\longrightarrow} 
    Z^{1/2} \bra{\beta; \outstate} a_{\outstate,g}^\dagger \ket{\alpha;\instate} \, ,\\
    i \int d^{\Dspace}x \, g(x)^* \dddt \bra{\beta; \outstate} \phi(x) \ket{\alpha;\instate}
    \stackrel{t\to +\infty}{\longrightarrow} 
    Z^{1/2} \bra{\beta; \outstate} a_{\outstate,g} \ket{\alpha;\instate} \, .
\end{align}

The S-matrix element for the scattering of $n$ incoming particles, with momentum wave 
functions $\psi_{g_i}(\mathbf{k})$, $i=1 \ldots n$, going into $m$ outgoing 
particles, with momentum wave functions $\psi_{g'_j}$, $j=1 \ldots m$, is 
\begin{equation}
    \label{eq:SMatrixNToM}
    S_{g'_1 \ldots g'_m, g_1 \ldots g_n} =
    \langle g'_1 \ldots g'_m; \outstate | g_1 \ldots g_n; \instate \rangle\, , 
\end{equation}
where we assume that the momentum wave functions have disjoint supports. As a consequence of 
the latter assumption
\begin{equation}
    a_{\outstate, g_1} \ket{g'_1 \ldots g'_m; \outstate} = 0\, ,
\end{equation}
and therefore
\begin{equation}
    \label{eq:ReductionStepOne}
    S_{g'_1 \ldots g'_m, g_1 \ldots g_n} =
    \langle g'_1 \ldots g'_m; \outstate |
    \left(a_{\instate,g_1}^\dagger - a_{\outstate, g_1}^\dagger\right)
    | g_2 \ldots g_n; \instate \rangle \, .
\end{equation}
Using the asymptotic condition, we can rewrite the matrix element using the smeared
Heisenberg field 
\begin{align}
    &S_{g'_1 \ldots g'_m, g_1 \ldots g_n} = 
    i Z^{1/2} \left(\lim_{t\to +\infty} - \lim_{t\to -\infty}\right) 
    \int d^{\Dspace}x \, g_1(x) \dddt 
    \bra{g'_1 \ldots g'_m; \outstate} \phi(x) \ket{g_2 \ldots g_n;\instate} 
    \nonumber \\
    & \quad = 
    i Z^{1/2} \int d^{\Dall}x\, \frac{\partial}{\partial t} 
    \left\{
        g_1(x) \dddt 
    \bra{g'_1 \ldots g'_m; \outstate} \phi(x) \ket{g_2 \ldots g_n;\instate}
    \right\} \\
    & \quad = 
    i Z^{1/2} \int d^{\Dall}x\, 
    g_1(x) \left(\partial_x^2 + m^2\right)  
    \bra{g'_1 \ldots g'_m; \outstate} \phi(x) \ket{g_2 \ldots g_n;\instate}
    \, .
\end{align}