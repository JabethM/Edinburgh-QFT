
\section{Path Integral for Gauge Fields}
\label{sec:gauge-fields}

The action for the gauge field, $A_\mu(x)$, can be written as
\begin{equation}
  \label{eq:GaugeAction}
  S[A] = \int d^Dx\, \left(-\frac14 F_{\mu\nu}(x)
    F^{\mu\nu}(x)\right)\, ,
\end{equation}
where $F_{\mu\nu}(x)=\partial_\mu A_\nu(x) - \partial_\nu A_\mu(x)$. 
Computing the variation of the action yields the classical equations
of motion, \ie Maxwell's equations in vacuum: 
\begin{equation}
  \label{eq:MaxEqs}
  \partial_\mu F^{\mu\nu}(x) = 0\, .
\end{equation}
Note that the action only depends on the field strength $F_\munu$,
and therefore is invariant under local {\em gauge transformations},
\begin{equation}
  \label{eq:GaugeTransf}
  A_\mu(x) \mapsto A^\Lambda_\mu(x) = A_\mu(x) + \partial_\mu \Lambda(x)\, .
\end{equation}
Other symmetries, like translation invariance and invariance under
Lorentz transformations are also encoded in
Eq.~(\ref{eq:GaugeAction}). 
The action is quadratic in the fields, and can be recast as 
\begin{equation}
  \label{eq:GaugeActionTwo}
  S[A] = \frac12 \int d^Dx\, A_\mu(x) \left[
    \partial^2 g^{\mu\nu} - \partial^\mu \partial^\nu
    \right] A_\nu(x)\, .
\end{equation}
It would be tempting at this stage to define the path integral by
analogy to the case of the scalar field:
\begin{equation}
  \label{eq:WrongPathInt}
  Z[J] = \int \mathcal{D}A \exp\left\{
    i \int d^Dx\, \left[
      -\frac14 F_{\mu\nu}(x) F^\munu(x) + J_\mu(x) A^\mu(x)\, .
      \right]
    \right\}
\end{equation}
Unfortunately, as discussed previously for the case of Gaussian
integrals and scalar fields, the integral above can be performed only
if the kernel is invertible. Writing the action in momentum space
yields the kernel in its diagonalised form, 
\begin{equation}
  \label{eq:GaugeActionMom}
  S[A] = \int \frac{d^Dk}{(2\pi)^D} \tilde{A}_\mu(k) \left[
    K^\munu(k)
    \right] \tilde{A}_\nu(-k)\, ,
\end{equation}
where the kernel is given by
\begin{equation}
  \label{eq:GaugeKernel}
  K^\munu(k) = -k^2 g^\munu + k^\mu k^\nu\, .
\end{equation}
It is clear from Eq.~(\ref{eq:GaugeActionMom}) that any longitudinal
component of the gauge field, 
\[
\tilde{A}_\mu(k) = k_\mu \tilde{\Lambda}(k)\, ,
\]
is an eigenfunction of the action kernel with vanishing
eigenvalue. Equivalently one could notice that the kernel is
proportional to the projector $\Pi^\munu$ on the transverse components
of the gauge field:
\begin{equation}
  \label{eq:GaugeProj}
  K^\munu(p) = -k^2 \Pi^\munu(k)\, .
\end{equation}
The action does not depend at all on the longitudinal components.
They are 'projected out' of the action, leaving a divergent integral
over a non compact domain. This is a direct consequence of the
redundancy in the usage of a four-vector to describe photons. There
are indeed four degrees of freedom in a real vector field, which is
used to represent photon with only {\em two} physical, transverse
polarizations. Using a four-vector allows an elegant implementation of
Lorentz covariance, but the price we pay is that we have unphysical
degrees of freedom in the action. The redundancy is at the origin of
the gauge symmetry of the action. Going back to position space, it is
easy to see that the longitudinal modes are {\em pure gauge} ones, \ie
$A_\mu(x)=\partial_\mu \Lambda(x)$. The solution to this problem is to
identify the redundant degrees of freedom, and factor out the
integration over these modes. 

\section{Faddeev-Popov procedure}
\label{sec:fadd-popov-proc}

A gauge {\em orbit} is a set of gauge configurations that are related
by gauge transformations: 
\begin{equation}
  \label{eq:GaugeOrbit}
  \Omega_A =\left\{
    A^\Lambda_\mu(x), \mathrm{for\ all}\ \Lambda(x)
    \right\}\, .
\end{equation}
As discussed above, for a given $A_\mu(x)$, all the field
configurations in $\Omega_A$ represent the same physical state, and
therefore a single representative should be included in the path
integral for each gauge orbit. We can select such representative by
requiring it to be the solution of a {\em gauge fixing} condition: 
\begin{equation}
  \label{eq:GaugeFixing}
  G\left(A_\mu(x)\right) = 0\, .
\end{equation}

\paragraph{A useful identity}

In order to insert a gauge fixing condition in the path intgral we are
going to make use of the following identity:
\begin{align} 
  \label{eq:FaddeevPopovOne}
  1 &= \int \mathcal{D}G\, \delta(G) = 
      \int \prod_x \Big[ dG(x)\, \delta\left(G(x)\right) \Big]\\
  \label{eq:FaddeevPopovTwo}
    &= \int \mathcal{D}\Lambda\, \delta\left(G\left(
      A^\Lambda_\mu\right)\right)
      \det \left(
      \frac{\delta G\left(A^\Lambda_\mu\right)}{\delta \Lambda}
      \right)\, .
\end{align}

\paragraph{Lorentz gauge}

As an example, the Lorentz gauge corresponds to the choice
\begin{equation}
  \label{eq:LorentzGauge}
  G\left(A_\mu(x)\right) = \partial_\mu A^\mu(x)\, ; 
\end{equation}
and therefore
\begin{align}
  G\left(A^\Lambda_\mu(x)\right)
  &= \partial_\mu\left(A^\mu(x) + \partial^\mu\Lambda(x)\right) \\
  &= \partial_\mu A^\mu(x) + \partial^2 \Lambda(x)\, .
\end{align}
In order to computer the Jacobian of the change of variables in
Eq.~(\ref{eq:FaddeevPopovTwo}), we need to consider
$G\left(A^\Lambda_\mu(x)\right)$ has a function of $\Lambda$, so that
\begin{equation}
  \label{eq:GaugeFixDeriv}
  \frac{\delta G\left(A^\Lambda_\mu(x)\right)}{\delta \Lambda(y)} = 
  \delta(x-y) \partial^2\, .
\end{equation}
In this simple case, we note that the functional derivative does not
depend on the gauge field $A_\mu$, and therefore we do not need to
work out the determinant in full detail. 

\paragraph{Gauge-fixed path integral}

We can now use Eq.~(\ref{eq:FaddeevPopovTwo}), and rewrite the
functional integral for a gauge theory as
\begin{align}
  \int \mathcal{D}A\, e^{i S[A]} 
  &= \det \left(
    \frac{\delta G\left(A^\lambda_\mu\right)}{\delta \Lambda}
    \right)\, \int \mathcal{D}\Lambda\, 
    \int \mathcal{D}A\,  e^{i S[A]} \, \delta\left(G\left(
      A^\Lambda_\mu\right)\right) \\
  \label{eq:GaugeFixedTwo}
  &\propto \int \mathcal{D}\Lambda\, 
    \int \mathcal{D}A^\Lambda\,  e^{i S[A^\Lambda]} \, \delta\left(G\left(
      A^\Lambda_\mu\right)\right) \, .
\end{align}
The integration measure is invariant under the transformation $A_\mu
\mapsto A_\mu + \partial_\mu \Lambda$, which is only a shift oof the
integration variables. The gauge invariance of the action means that
$S[A] = S[A^\Lambda]$. Finally we can rename the integration variable,
and rewrite Eq.~(\ref{eq:GaugeFixedTwo})
\begin{equation}
  \label{eq:GaugeFixedThree}
  \int \mathcal{D}A\, e^{i S[A]} 
  \propto \int \mathcal{D}\Lambda\, 
    \int \mathcal{D}A\,  e^{i S[A]} \, \delta\left(G\left(
      A_\mu\right)\right) \, .
\end{equation}
We have therefore achieved our goal, namely to separate the
integration over the gauge copies, which is now factored out in front
of the path integral. The remaining integration is over the gauge
fields but includes the gauge-fixing delta function, which selects one
representative for each gauge orbit. This procedure is called {\em
  Faddeev-Popov} method, and turns out to be particular simple for the
U(1) theory, where the Jacobian turns out to be independent of the
gauge field, and drops out of the integral. The procedure yields a
more interesting result for the case of non-Abelian gauge symmetry.  

\paragraph{Generalised gauge }

We can generalise the gauge condition considering
\begin{equation}
  \label{eq:GenGauge}
  G\left(A_\mu(x)\right) = \partial_\mu A^\mu(x) - \omega(x)\, ,
\end{equation}
where $\omega$ is a generic function. Using the Faddeev-Popov trick,
we can write the path integral as
\begin{equation}
  \label{eq:GenGaugeOne}
  \int \mathcal{D}A\, e^{i S[A]} 
  \propto \int \mathcal{D}\Lambda\, 
    \int \mathcal{D}A\,  e^{i S[A]} \, \delta\Big(
        \partial_\mu A^\mu(x) - \omega(x)
      \Big) \, .
\end{equation}
Eq.~(\ref{eq:GenGaugeOne}) can be integrated over $\omega$ with a
Gaussian weight:
\begin{align}
   \int \mathcal{D}A\, e^{i S[A]} 
  &\propto \int \mathcal{D}\omega\, \exp \left[
    -i \int d^Dx\, \frac{\omega(x)^2}{2\xi}
    \right]\, 
    \int \mathcal{D}\Lambda\, 
    \int \mathcal{D}A\,  e^{i S[A]} \, \delta\Big(
        \partial_\mu A^\mu(x) - \omega(x)
      \Big) \nonumber \\
  &= \left(\int \mathcal{D}\Lambda\right)\, 
    \int \mathcal{D}A\,  e^{i S[A]} \, 
    \exp \left[
    -i \int d^Dx\, \frac{1}{2\xi}
    \left(\partial_\mu A^\mu(x)\right)^2
    \right]\, .
\end{align}
Once again we have factored out the volume of the gauge orbit, but now
instead of a gauge fixing delta function in the path integral, we have
a non-trivial weight for the longitudinal modes in the
exponential. The modified action can be written as:
 \begin{align}
   S[A] &- \frac{1}{2\xi} \int d^Dx\, \left(
          \partial_\mu A^\mu(x) 
          \right)^2 = \nonumber \\
   &= \frac12 \int d^Dx\, 
     A_\mu(x) \left[
     \partial^2 g^\munu -
     \left(1-\frac{1}{\xi}\right) \partial^\mu\partial^\nu 
     \right] A_\nu(x)\, .
 \end{align}
The new kernel in momentum space is
\begin{align}
  K_\xi^\munu(k) =-k^2 g^\munu + \left(1-\frac{1}{\xi}\right) k^\mu
  k^\nu\, ,
\end{align}
and it can be readily checked that the longitudinal modes are no
longer zero modes of $K_\xi$, which is now invertible. Its inverse,
which we denote $\tilde{D}_F^\munu(k)$ is the Feynman propagator for
the photon field: 
\begin{align}
  \tilde{D}_F^\munu(k) = \frac{-i}{k^2+i\epsilon} 
  \left[
  g^\munu - \left(1-\xi\right)\frac{k^\mu k^\nu}{k^2}
  \right]\, .
\end{align}
The choices $\xi=0$ and $\xi=1$ are called respectively the Landau and
Feynman gauge propagators.

The Gaussian integral for the free theory can now be performed, 
\begin{align}
  Z_0[J] = \exp \Big[
  \frac12 \int \frac{d^Dk}{(2\pi)^D}\, \tilde{J}_\mu(k) 
  \tilde{D}^\munu_F(k) J_\nu(-k)
  \Big]\, .
\end{align}
The propagator in momentum space is obtained by Fourier transforming
the expresssion in momentum space, 
\begin{align}
  D^\munu_f(x-y) 
  &= \int \frac{d^Dk}{(2\pi)^D}\,
    e^{-i k\cdot (x-y)} \tilde{D}^\munu_F(k) \\
  &= 
    \begin{tikzpicture}[baseline={([yshift=1.4ex]current bounding box.center)}]
      \begin{feynman}[inline=(a)]
        \vertex (a);
        \vertex (b);
        \diagram {
          a -- [photon] b,
        };
        \vertex [below=0.2em of a] {\(_{x}\)};  
        \vertex [below=0.2em of b] {\(_{y}\)};  
      \end{feynman}
    \end{tikzpicture}\, .
\end{align}
The photon propagator is usually represented by a wavy line, as shown
in the second line above. 
