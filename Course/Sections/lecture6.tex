\newcommand{\tphi}{\tilde{\phi}}
\newcommand{\tj}{\tilde{J}}
\newcommand{\tchi}{\tilde{\chi}}
\newcommand{\psibar}{\bar{\psi}}
\newcommand{\etabar}{\bar{\eta}}
\newcommand{\munu}{{\mu\nu}}
\newcommand{\tDelta}{\tilde{\Delta}}
\newcommand{\SProp}[1]{\frac{1}{#1^2-m^2+i\epsilon}}


\section{A first look at divergences}
\label{sec:a-first-look}

In this lecture we will aim to develop a self-consistent treatment of
divergences in QFT. This is a vast topic, which can hardly be
addressed exhaustively in the time that we have. Therefore we will
follow the following steps.
\begin{enumerate}
\item Compute the scalar two-point function beyond the first order in
  perturbation theory. As we try to perform this calculation we will
  encounter our first divergent integral.
\item Discuss the regularization of divergencies; \ie a procedure that
  allows us to manipulate well-defined mathematical expressions, and
  to identify the structure of the divergencies.
\item Discuss the renormalization of divergencies; \ie the conditions
  that are necessary for a quantum field theory to produce finite,
  unambiguous predictions. 
\end{enumerate}

\subsection{Scalar two-point function}
\label{sec:scalar-two-point}

Working in perturbation theory, we compute the two-point function
\begin{equation}
  \label{eq:TwoPtMom}
  \tilde{G}^{(2)}\left(p,p'\right) =
  (2\pi)^D \delta\left(p+p'\right) \frac{1}{i} \tilde{\Delta}_F(p)\, ,
\end{equation}
as a Taylor expansion in powers of the coupling constant
\begin{equation}
  \label{eq:TwoPtMomPert}
  \tilde{G}^{(2)}\left(p,p'\right) = \sum_k g^k
  \tilde{G}^{(2,k)}\left(p,p'\right)\, .
\end{equation}
As discussed before, the delta function in Eq.~\ref{eq:TwoPtMom}
ensures momentum conservation. For all practical purposes, we should
remember that it is there, and work on the perturbative expansion of
the full propagator $\tilde{\Delta}_F(p)$:
\begin{equation}
  \label{eq:PropMomPert}
  \tilde{\Delta}_F(p) = \sum_k g^k \tilde{\Delta}^{(k)}_F(p)\, .
\end{equation}
From our previous computations
\begin{align}
  g^2 \frac{1}{i} \tDelta_F^{(2)}(p)
  &= -\frac{g^2}{2} \SProp{p}
\end{align}
