\documentclass[12pt,a4paper]{article}
%
\setlength{\oddsidemargin}{  0mm}
\setlength{\topmargin}    { -12mm}
\setlength{\textheight}   { 246mm}
\setlength{\textwidth}    { 165mm}
\setlength{\parindent}    {  0   pt}  % not actually required but they

\setlength{\parskip}      {  6   pt}  % make paragraphs look less ugly

%

\usepackage{amsmath}
\usepackage{amssymb}
\usepackage{slashed}
\pagestyle{empty}

\pagestyle{headings}

\textwidth=440pt
\hoffset=-0.6truein

\usepackage{amsmath}
\usepackage{amsfonts}
\usepackage{amssymb}
%\usepackage{amsthm}
\usepackage{dsfont}
\usepackage{pifont}
%\usepackage{bbold}
\usepackage{graphicx}
\usepackage{epstopdf}
\usepackage{epsfig}
%\usepackage{bibunits}
%\usepackage{theorem}
\usepackage[framed]{ntheorem}
\usepackage{framed}
%\usepackage{showlabels}
\usepackage{makeidx}
\usepackage{simplewick}
\usepackage{tikz-feynman}
\usepackage{slashed}
\usepackage{appendix}

\tikzfeynmanset{compat=1.0.0}  

\newcommand{\tick}{\ding{52}}
\newcommand{\notick}{\ding{56}}
\newcommand{\D}{\displaystyle}
\newcommand{\mphys}{m_\mathrm{phys}}
\newcommand{\tphi}{\tilde{\phi}}
\newcommand{\tj}{\tilde{J}}
\newcommand{\tchi}{\tilde{\chi}}
\newcommand{\psibar}{\bar{\psi}}
\newcommand{\etabar}{\bar{\eta}}
\newcommand{\munu}{{\mu\nu}}
\renewcommand{\chaptername}{Lecture}

\def\bfx{{\mathbf x}}
\def\bfxp{{\mathbf x^\prime}}
\def\bfy{{\mathbf y}}
\def\bfyp{{\mathbf y^\prime}}
\def\bfp{{\mathbf p}}
\def\bfpp{{\mathbf p^\prime}}
\def\ddt{\frac{d}{dt}}
\def\ddtt{\frac{d^2}{dt^2}}
\def\ie{{\it i.e.}\ }
\def\eg{{\it e.g.}\ }
\def\viz{{\it viz.}\ }
\def\matF{\mathcal F}
\def\matE{\mathcal E}
\def\GL{\mathrm{GL}}
\def\kpsi{|\psi\rangle}
\def\kpsione{|\psi_1\rangle}
\def\kpsitwo{|\psi_2\rangle}
\def\kpsionep{|\psi_1^\prime\rangle}
\def\kpsitwop{|\psi_2^\prime\rangle}
\def\kpsii{|\psi_i\rangle}
\def\kpsin{|\psi_n\rangle}
\def\kpsip{|\psi^\prime\rangle}
\def\bpsi{\langle\psi |}
\def\bpsione{\langle\psi_1 |}
\def\bpsitwo{\langle\psi_2 |}
\def\bpsii{\langle\psi_i |}
\def\bpsip{\langle\psi^\prime |}
\def\kphi{|\phi\rangle}
\def\kphione{|\phi_1\rangle}
\def\kphitwo{|\phi_2\rangle}
\def\kphii{|\phi_i\rangle}
\def\kphip{|\phi^\prime\rangle}
\def\bphi{\langle\phi |}
\def\bphione{\langle\phi_1 |}
\def\bphitwo{\langle\phi_2 |}
\def\bphii{\langle\phi_i |}
\def\bphip{\langle\phi^\prime |}
\def\bchi{\langle\chi |}
\def\bchione{\langle\chi_1 |}
\def\bchitwo{\langle\chi_2 |}
\def\bchii{\langle\chi_i |}
\def\bchip{\langle\chi^\prime |}
\def\kjm{|j,m\rangle}
\def\tr{\mathrm{Tr}}
\def\Rn{\mathbb{R}^n}
\def\Cn{\mathbb{C}^n}
\def\id{\mathds{1}}
{\theoremstyle{plain} \theorembodyfont{\rmfamily} \newframedtheorem{Ex}{Exercise}[section]}
{\theoremstyle{plain} \theorembodyfont{\rmfamily} \newframedtheorem{Def}{Definition}[section]}
{\theoremstyle{plain} \theorembodyfont{\rmfamily} \newframedtheorem{Thm}{Theorem}[section]}

\newcommand{\clearemptydoublepage}{\newpage{\pagestyle{empty}\cleardoublepage}}
\newcommand{\HRule}{\rule{\linewidth}{0.5mm}}
\newcommand{\iu}{\underline{i}}
\newcommand{\ju}{\underline{j}}
\newcommand{\ku}{\underline{k}}
\newcommand{\ru}{\underline{r}}
\newcommand{\pu}{\underline{p}}
\newcommand{\Lu}{\underline{L}}
\newcommand{\Ju}{\underline{J}}
\newcommand{\lap}{\nabla^2}
\newcommand{\ad}{\hat{a}}
\newcommand{\ac}{\hat{a}^\dagger}
\newcommand{\re}{\mathrm{Re}}
\newcommand{\ket}[1]{| #1 \rangle}
\newcommand{\bra}[1]{\langle #1 |}
\newcommand{\braket}[2]{\langle #1 | #2 \rangle}
\newcommand{\pref}[1]{(\ref{#1})}
\newcommand{\Eqref}[1]{Eq.~(\ref{#1})}
\newcommand{\del}{\v{\nabla}}				% Underlined del



\newcommand{\tphi}{\tilde{\phi}}
\newcommand{\tj}{\tilde{J}}
\newcommand{\tchi}{\tilde{\chi}}
\newcommand{\psibar}{\bar{\psi}}
\newcommand{\etabar}{\bar{\eta}}
\newcommand{\munu}{{\mu\nu}}

\begin{document}
\begin{center}
{\bf Quantum Field Theory}\\[\baselineskip]
\end{center}
{\bf Problem Sheet 5}

\begin{enumerate}
  \item {\it Translation Ward identity} \\
    
    Find the variation of the action for the free scalar field under
    the field transformation
    \[
    \phi(x) \mapsto \phi'(x) = \phi(x) - a(x) \partial_\mu \phi(x)\, .
    \]
    Deduce the Ward identities generated by translation invariance. 

    \bigskip
    
  \item {\it Grassmann integrals}\\

    Integrals over Grassmann variables are defined by specifying two
    {\em operational} rules:
    \begin{align}
      \int d\psi_\alpha &= 0\, , \nonumber \\
      \int d\psi_\alpha \psi_\beta &= \delta_{\alpha\beta}\, . \nonumber
    \end{align}
    Briefly discuss why this is the case.

    Show that, for an $N\times N$ matrix $A_{\alpha\beta}$ 
    \[
    \int \prod_{\beta=1}^N d\psi_\beta\,
    \prod_{\alpha=1}^N d\psibar_\alpha\,
    \exp\left(
      \psibar_\alpha A_{\alpha\beta} \psi_\beta
      \right) = \det A
    \]
    
    {\em Hint:} it is useful to remember that 
    \[
    \det A = \sum_{\beta_1\ldots \beta_N} 
    \epsilon_{\beta_1\ldots\beta_N} 
    A_{1\beta_1} \ldots A_{N\beta_N}\, .
    \]

    \bigskip

    \item {\it Dirac propagator} \\
      
      Prove that the Dirac propagator is the inverse of the kinetic
      term in the action, \ie
      \[
      \left(i \slashed{\partial}_x -m \right)_{\alpha\beta}
      S_{\beta\gamma}(x-y) = i \delta(x-y) \delta_{\alpha\gamma}\, .
      \]

  \item {\it LSZ reduction for fermions}\\
    
    For the case of fermions the operator $\psi(x)$ can be decomposed
    as
    \[
    \psi(x) = \int d\Omega_p\, \sum_{s=\pm1/2} \left[
      e^{-i p\cdot x} a(\mathbf{p},s) u(\mathbf{p},s) + 
      e^{i p\cdot x} b^\dagger(\mathbf{p},s) v(\mathbf{p},s)
      \right]\, .
    \]
    This relation can be inverted, yielding:
    \begin{align}
%      \label{eq:adag}
      a^\dagger(\mathbf{p},s) = \int d^3x\, e^{-ip\cdot x}
      \bar{\psi}(x) \gamma^0 u(\mathbf{p},s) \, , \nonumber \\
%      \label{eq:bdag}
      b^\dagger(\mathbf{p},s) = \int d^3x\, e^{-ip\cdot x}
      \bar{v}(\mathbf{p},s) \gamma^0 \psi(x) \, . \nonumber 
    \end{align}
    Following the same reasoning that we used in the case of a scalar
    field, let us introduce in the interacting theory time-dependent
    creation/annihilation operators for fermions and antifermions
    according to the expressions above. 
    Show that
    \begin{align}
      \label{eq:3}
      a^\dagger(+\infty,\mathbf{p},s) - 
      a^\dagger(-\infty,\mathbf{p},s) = 
      \int d^4x\,
      e^{-ip\cdot x} \bar\psi(x)
      (i\overleftarrow{\slashed{\partial}}+m) u(\mathbf{p},s)\, ,
    \end{align}
    and similar relations for the other creation/annihilation
    operators. 
 
    The scattering amplitude for a $2\longrightarrow 2$ process can be
    written as: 
    \begin{align}
      \label{eq:4}
      \langle p_1', s'; p_2', r';\mathrm{out} | 
      &p_1, s; p_2, r; \mathrm{in} \rangle 
      = \nonumber \\
      = \langle 0 | T\left(
        b(+\infty,\mathbf{p}_2',r') a(+\infty,\mathbf{p}_1',s') 
        a^\dagger(-\infty,\mathbf{p}_1,s) b^\dagger(-\infty,\mathbf{p}_2,r)
        \right) |0 \rangle\, .
    \end{align}
    Show that
    \begin{align}
      \label{eq:5}
      \langle p_1', s'; p_2', r';\mathrm{out} |
      & p_1, s; p_2, r; \mathrm{in} \rangle = (-i)^{2} (i)^{2} \nonumber \\
      & \times \int d^4x_1\, e^{-i p_1\cdot x_1}  
        \int d^4x_2\, e^{-i p_2\cdot x_2}
        \int d^4x_1'\, e^{i p_1'\cdot x_1'} 
        \int d^4x_2'\, e^{i k_2'\cdot x_2'} \nonumber \\
      & \times \left[\bar{u}(\mathbf{p}_1',s') 
         \left(i\overrightarrow{\slashed{\partial}}_{x_1'} -
         m\right)\right]_{\alpha_1}\,  
         \left[\bar{v}(\mathbf{p}_2,r) 
         \left(i\overrightarrow{\slashed{\partial}}_{x_2} -
         m\right)\right]_{\beta_2}   \nonumber \\
      & \times \langle 0 | T\left(
        \bar\psi_{\alpha_1}(x_1') \psi_{\alpha_1}(x_2') \bar{\psi}_{\beta_1}(x_1) \psi_{\beta_2}(x_2)
        \right) |0 \rangle \nonumber \\
      & \times \left[\left(-i\overleftarrow{\slashed{\partial}}_{x_1} -
        m\right) u(\mathbf{p}_1,s)\right]_{\beta_1}\,  
        \left[\left(-i\overleftarrow{\slashed{\partial}}_{x_2'} -
        m\right) v(\mathbf{p}_2',r')\right]_{\alpha_2}\, .        
    \end{align}

\end{enumerate}

\vfill
\hspace*{\fill}\tiny L Del Debbio, October 2018.
\end{document}
